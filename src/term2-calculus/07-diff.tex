\Subsection{Дифференцируемость функции многих переменных}
\begin{definition}
    $f\!: E \subset \R^n \to R^m$,  $a \in \Int E$.

     $f$ --- дифференцируема в точке  $a$, если  $f(a+h) = f(a) + Th + o(\|h\|)$ при  $h \to \overrightarrow{0}$, где  $T\!: \R^n \to \R^m$ --- линейное отображение.
\end{definition}
\begin{remark}
    Если $f$ дифференцируема в точке  $a$, то  $T$ определена однозначно. 
\end{remark}
\begin{proof}
    $f(a+th) = f(a) + T(th) + o(\|th\|) = f(a) + tTh + o(t\|h\|) = f(a) + tTh + o(t)$. $\|h\|$ --- константа, поэтому можно выкинуть.

     $Th = \lim\limits_{t \to 0} \frac{f(a+th) - f(a)}{t}$
\end{proof}
\begin{remark}
    Если $f$ дифференцируем в точке  $a$, то  $f$ непрерывна в точке  $a$.

     $f(a+h) = f(a) + Th + o(\|h\|) \xrightarrow{h \to 0} f(a)$.
\end{remark}
\begin{definition}
    $T$ --- дифференциал $f$ в точке  $a$. Обозначается  $\mathrm{d}_af$. 
\end{definition}
\begin{definition}
    Матрицу отображения $\mathrm{d}_af$ назовем матрицей Якоби  $f$ в точке  $a$.
\end{definition}

Важный частный случай $m = 1$.  $f(a+h) = f(a) + \langle v, h \rangle o(\| h\|)$. $V$ --- градиент функции  $f$ в точке  $a$. Обозначается  $\text{grad}\ f$,  $\nabla f(a)$ (набла).

\begin{theorem}
Пусть $f = \begin{pmatrix} f_1 \\ f_2 \\ \vdots \\ f_n \end{pmatrix}\!: E \to \R^m$, $a \in \Int E$.

Тогда $f$ --- дифференцируема в точке  $a \iff f_j$ дифференцируема в точке  $a \forall j$.
\end{theorem} 
\begin{proof}
    $f(a+h) = f(a) + Th + \alpha(h)$, где  $\frac{\alpha(h)}{\|h\|} \xrightarrow{h \to 0} 0$.
    \begin{itemize}
        \item $\Rightarrow$.  $f_j(a+h) = f_j(a) + (Th)_j + \alpha_j(h)$. Надо доказать, что  $\frac{\alpha_j(h)}{\|h\|} \to 0$. $|\alpha_j(h)| \le \|\alpha(h)\|$. Значит, $\frac{|\alpha_j(h)|}{\|h\|} \le \frac{\|\alpha_j(h)\|}{\|h\|} \to 0$.
        \item $\Leftarrow$. Знаем, что  $f_j(a+h) = f_j(a) + T_jh + \alpha_j(h)$, где  $\frac{\alpha_j(h)}{\|h\|} \to 0$. 

            $\alpha(h) = \begin{pmatrix} \alpha_1(h) \\ \vdots \\ \alpha_m(h) \end{pmatrix}$. Надо доказать, что $\frac{\|\alpha(h)\|}{\|h\|} \to 0$.

            Заметим, что $\frac{\|\alpha(h)\|}{\|h\|} = \frac{\sqrt{\alpha_1(h)^2 + \ldots + \alpha_m(h)^2}}{\|h\|} \le  \frac{|\alpha_1(h)| + \ldots + |\alpha_m(h)|}{\|h\|} \to 0$.
    \end{itemize}
\end{proof}
\begin{consequence}
    Строки матрицы Якоби --- градиенты координатных функций
\end{consequence}
\begin{proof}
    $Th = \begin{pmatrix} T_1 \\ T_2 \\ \vdots \\ T_m \end{pmatrix}(h) = \begin{pmatrix} T_1h \\ \vdots \\ T_mh \end{pmatrix}$.
\end{proof}
\begin{definition}
    Производная по направлению. $f\!: E \to \R, \|h\|=1$.

     $\frac{\partial f}{\partial h}(a) \coloneqq \lim\limits_{t \to 0} \frac{f(a+th) - f(a)}{t}$.
\end{definition}
\begin{remark}
    $\frac{\partial f}{\partial h}(a) = \mathrm{d}_af(h)$.
\end{remark}
\begin{theorem}[экстремальное свойство градиента]
    $\left| \frac{\partial f}{\partial h}(a) \right| \le \|\nabla f(a)\|$, причем равенство достигается при $h = \pm \frac{\nabla f(a)}{\|\nabla f(a)\|}$.
\end{theorem}
\begin{proof}
    $\frac{\partial f}{\partial h}(a) = \mathrm{d}_af(h) = \langle \nabla f(a), h \rangle$.
\\
    $\left| \frac{\partial f}{\partial h}(a) \right| = | \langle \nabla f(a), h| \le \|\nabla f(a) \| \cdot \|h\| = \|\nabla f(a)\|$.

    В неравенстве Коши-Буняковского равенство $\iff$ векторы пропорциональны  $\implies h = \pm \frac{\nabla f(a)}{\|\nabla f(a)\|}$.
\end{proof}
\begin{definition}
    Частные производные. $f\!: E \to \R$ (это $\iff f$ --- скалярная), $e_k$ --- базисный вектор (везде нули кроме $k$-й позиции).

     $\frac{\partial f}{\partial x_k}, f'_{x_k}, D_{x_k}f,\ldots$. $\frac{\partial f}{\partial x_k} (a) \coloneqq \frac{\partial f}{\partial e_k}(a)$.
\end{definition}
\begin{remark}
    $f(x_1, x_2, \ldots, x_n)$. $\frac{\partial f}{\partial x_1}(a) = \lim\limits_{t \to 0} \frac{f(a_1 + t, a_2, \ldots,a_n) - f(a_1, a_2,\ldots, a_n)}{t} = g'(a_1)$, где $g(s) \coloneqq f(s, a_2, \ldots, a_n)$.
\end{remark}
\begin{example}
    $f(x, y) = x^y$.  $\frac{\partial f}{\partial x} = yx^{y-1}$, $\frac{\partial f}{\partial y} = x^y \cdot \ln x$.
\end{example}
\begin{theorem}
    $\nabla f(a) = \left( \frac{\partial f}{\partial x_1}(a), \frac{\partial f}{\partial x_2}(a), \ldots, \frac{\partial f}{\partial x_n}(a) \right)$.
\end{theorem}
\begin{proof}
   $\frac{\partial f}{\partial h} = \langle \nabla f(a), h \rangle$.

   $\frac{\partial f}{\partial x_k}(a) = \frac{\partial f}{\partial e_k}(a) = \langle \nabla f(a), e_k \rangle = (\nabla f(a))_k$.
\end{proof}
\begin{consequence}
    $f\!: E \subset \R^n \to \R^m$,  $a \in \Int E$,  $f$ --- дифференцируема в точке  $a$.

    Тогда  $f'(a) = \begin{pmatrix} \frac{\partial f_1}{x_1} & \frac{\partial f_1}{x_2} & \ldots & \frac{\partial f_1}{x_n}\\  \frac{\partial f_2}{x_1} &\frac{\partial f_2}{x_2} & \ldots &\frac{\partial f_2}{x_n}\\ \ldots & \ldots & \ldots & \ldots\\ \frac{\partial f_m}{x_1} & \frac{\partial f_m}{x_2} & \ldots & \frac{\partial f_m}{x_n} \end{pmatrix}$.
\end{consequence}
\begin{theorem}(линейность дифференциала)
    $f, g\!: E \subset \R^n \to \R^m$,  $a \in \Int E$,  $\lambda \in \R$, $f, g$ дифференцируемы в точке $a$.

    Тогда $f + g, \lambda f$ --- дифференцируемы в  $a$ и $\mathrm{d}_a(f+g) = \mathrm{d}_af + \mathrm{d}_ag $ и $\mathrm{d}_a(\lambda f) = \lambda \mathrm{d}_a f$.
\end{theorem}
\begin{proof}
    $f(a+h) = f(a) + \mathrm{d}_af(h) + \alpha(h)$, $\frac{\alpha(h)}{\|h\|} \to 0$, $g(a + h) = g(a) + \mathrm{d}_ag(h) + \beta(h)$, $\frac{\beta(h)}{\|h\|} \to 0$.

    Тогда $f(a+h) + g(a+h) = f(a) + g(a) + \mathrm{d}_af(h) + \mathrm{d}_ag(h) + \alpha(h) + \beta(h)$. Считаем, что  $\alpha(h) + \beta(h) = o(\|h\|)$.

     $\lambda f(a+h) = \lambda f(a) + \lambda \mathrm{d_a}f(h) + \lambda\alpha(h)$.
\end{proof}
\begin{theorem}[дифференцируемость композиции]
    $f\!: E \subset \R^n \to D \subset \R^m$, $g\!: D \subset \R^m \to \R^l$,  $a \in \Int E, b = f(a) \in \Int D$.

    Тогда, если  $f$ дифференцируема в  $a$,  $g$ дифференцируема в  $b = f(a)$, то  $g \circ f$ дифференцируема в точке  $a$ и  $\mathrm{d}_a(g \circ f) = \mathrm{d}_{f(a)}g \circ \mathrm{d}_af$.
\end{theorem}
\begin{proof}
    $f(a+h) = \underbracket{f(a)}_{=b} + \underbracket{\mathrm{d}_af(h) + \alpha(h)}_{=k}$, где  $\frac{\alpha(h)}{\|h\|} \xrightarrow{h \to 0} 0$.

    $g(b+k) = g(b) + \mathrm{d}_bg(k) + \beta(k)$, где  $\frac{\beta(k)}{\|k\|} \xrightarrow{k \to 0} 0$.

    $g \circ f(a + h) = g(b + k) = g(b) + \underbracket{\mathrm{d}_bg(\mathrm{d}_af(h)+\alpha(h))}_{\mathrm{d}_bg(\mathrm{d_af(h)}) + \mathrm{d}_bg(\alpha(h))} = g \circ f(a) + \mathrm{d}_bg \circ \mathrm{d}_af(h) + \mathrm{d}_bg(\alpha(h)) + \beta(k)$.

    Хотим показать, что все корректно.

    $\frac{\|\mathrm{d}_bg(\alpha(h))\|}{\|h\|} \le \|\mathrm{d}_b(g)\| \underbracket{\frac{\|\alpha(h)\|}{\|h\|}}_{\to 0}$. $k = \mathrm{d}_af(h) + \alpha(h)$.  $\|k\| \le \|\mathrm{d}_af(h)\| + \|\alpha(h)\| \le \|\mathrm{d}_af\| \cdot \|h\| + \|\alpha(h)\| \to 0$, так как $\frac{\|k\|}{\|h\|} \le \|\mathrm{d}_af\| + \frac{\|\alpha(h)\|}{\|h\|}$.

    В итоге, $\frac{\|\beta(k)\|}{\|k\|} \cdot \frac{\|k\|}{\|h\|} \xrightarrow{h \to 0} 0$.
\end{proof}
\begin{consequence}
    $(g \circ f)'(a) = g'(f(a)) f'(a)$.
\end{consequence}
\begin{theorem}[Дифферециал произведения скалярной и векторной функции]
    $E \subset \R^n, a \in \Int E$,  $\lambda\!: E \to \R, f\!: E \to \R^m$,  $\lambda$ и  $f$ дифференцируемы в точке  $a$. Тогда  $\lambda f$ дифференцируема в точке  $a$ и  $\mathrm{d}_a(\lambda f)(h) = \mathrm{d}_a \lambda(h) f(a) + \lambda(a) \cdot \mathrm{d}_af(h)$.
\end{theorem}
\begin{proof}
    $f(a+h) = f(a) + \mathrm{d}_af(h) + \alpha(h)$,  $\frac{\alpha(h)}{\|h\|} \to 0$. $\lambda(a + h) = \lambda(a) + \mathrm{d}_a\lambda(h) + \beta(h), \frac{\beta(h)}{\|h\|} \to 0$.

    $\lambda(a+h)f(a+h) = \lambda(a)f(a) + {\color{orange}\mathrm{d}_a\lambda(h)f(a) + \lambda(a)\mathrm{d}_af(h)} + {\color{purple}\mathrm{d}_a\lambda(h)\cdot \mathrm{d}_af(h)} + \lambda(a) \cdot \alpha(h) + f(a) \beta(h) + {\color{blue}\mathrm{d}_af(h)\beta(h)} + {\color{olive}\mathrm{d}_a\lambda(h)\alpha(h)} +\alpha(h)\beta(h)$. 

    Заметим, что {\color{orange}второе и третье} слагаемые очевидно подходят под наше свойство. Теперь заметим, что  {\color{purple}$\text{const} \cdot \|h\|^2 = o(\|h\|)$} и  {\color{blue}$\text{const} \cdot \|h\| \beta(h) = o(\|h\|)$} и  {\color{olive}$\text{const} \|h\| \|\alpha(h)\| = o(\|h\|)$}. И всё получается. 
\end{proof}
\begin{theorem}[о дифференциале скалярного произведения]
    $f, g\!: E \to \R^m, a \in \Int E$,  $f, g$ --- дифференцируемы в  $a$. 

    Тогда  $\langle f, g\rangle$ дифференцируемы в  $a$. $\mathrm{d}_a \langle f, g \rangle(h) = \langle \mathrm{d}_af(h), g(a) \rangle + \langle f(a), \mathrm{d}_ag(h) \rangle$.
\end{theorem}
\begin{proof}
    $\langle f(x), g(x) = \sum\limits_{j=1}^m f_j(x) g_j(x)$.

     $\mathrm{d}_a\langle f, g \rangle(h) = \sum\limits_{j=1}^m \mathrm{d}_a(f_jg_j)(h) = \sum\limits_{j=1}^m (\mathrm{d}_a g_j(h)f_j(a) + g_j(a) \mathrm{d}_af_j(h)) = \sum\limits_{j=1}^m f_j(a) \mathrm{d}_ag_j(h) + \sum\limits_{j=1}^m \mathrm{d}_a f_j(h)g_j(a) = \langle f(a), \mathrm{d}_ag(h)\rangle + \langle \mathrm{d}_a f(h), g(a) \rangle$.
\end{proof}
\begin{remark}
    При $n=1$  $\langle f, g \rangle'(a) = \langle f'(a), g(a)\rangle + \langle f(a), g'(a) \rangle$.
\end{remark}
\begin{theorem}[Лагранжа для векторнозначных функций]
    $f\!: [a, b] \to \R^m$,  $f$ --- дифференцируема во всех точках из  $a, b$ и непрерывна на  $[a, b]$.

    Тогда существует  $c \in (a, b)$, такая что  $\|f(b) - f(a)\| \le \|f'(c)\|(b-a)$.
\end{theorem}
\begin{proof}
    $\vphi(t) \coloneqq \langle f(t), f(b) - f(a) \rangle$ --- дифференцируемая функция  $\implies \exists c \in (a, b)\!: \|f(b) - f(a)\|^2 = \langle f(b), f(b) - f(a) \rangle - \langle f(a), f(b) - f(a) \rangle = \vphi(b) - \vphi(a)= \vphi'(c)(b-a) = (b-a) \langle f'(c), f(b) - f(a) \rangle \le (b-a) \|f'(c)\|\|f(b) - f(a)\|$.\\
    $\vphi'(t) = \langle f'(t), f(b) - f(a) \rangle$.
\end{proof}
\begin{example}
    $m = 2$,  $[a, b] = [0, 2\pi], f(t) = \begin{pmatrix} \cos t \\ \sin t \end{pmatrix}, f'(t) = \begin{pmatrix} -\sin t \\ \cos t \end{pmatrix}$. $f(2\pi) - f(0) = \overrightarrow{0}$,  $\|f'(t)\| = 1$. Тогда получаем  $\|f(2\pi) - f(0)\| = 0 < 2\pi\|f'(c)\|$.
\end{example}
\Subsection{Непрерывная дифференцируемость}
\begin{theorem}
    $f\!: E \subset \R^n \to \R$,  $a \in \Int E$. Все частные производные функции  $f$ существуют в окрестности  $a$ и непрерывны в ней. Тогда  $f$ дифференцируема в точке  $a$.
\end{theorem}
\begin{proof}
    $f(a+h) - f(a) = \sum\limits_{i=1}^n \frac{\partial f}{x_i}(a) h_i + o(\|h\|)$, $R(h) = f(a + h) - f(a) - \sum\limits_{i=1}^n \frac{\partial f}{\partial x_i}(a) h_i$.

    $b_0 = a, b_1 = (a_1 + h, a_2, \ldots, a_n), b_k = (a_1 + h_1, a_2 + h_2, \ldots, a_k + h_k, a_{k+1}, \ldots, a_n).$

    $f(b_k) - f(b_{k-1}) = f(a_1 + h_1, \ldots, a_{k-1} + h_{k-1}, a_k + h_k, a_{k+1}, \ldots, a_n) - f(a_1 + h_1, \ldots, a_{k-1} + h_{k-1}, a_k, \ldots, a_n) = h_k \frac{\partial f}{\partial x_k}(b_{k-1} - \Theta_k h_k e_k)$ ($0 < \Theta_k < 1$). 

    Тогда $f(a + h) - f(a) = f(b_n) - f(b_0) = \sum\limits_{k=1}^n h_k \frac{\partial f}{\partial x_k}(b_{k-1} + \Theta_k h_ke_k) = \sum\limits_{k=1}^n h_k \frac{\partial f}{\partial x_k}(a) + \sum\limits_{k=1}^nh_k(\frac{\partial f}{\partial x_k}(b_{k-1} + \Theta_k h_k e_k) - \frac{\partial f}{\partial x_k}(a))$.

    Тогда $|R(h)| = \left|\sum\limits_{k=1}^n h_k \left( \frac{\partial f}{\partial x_k}(b_{k-1} + \Theta_k h_k e_k) - \frac{\partial f}{\partial x_k}(a)\right)\right| \le \|h\| (\sum(...)^2)^{\frac{1}{2}}$, а $(\sum(...)^2)^{\frac{1}{2}} \to 0$.
\end{proof}
