Просто аксиом и утверждений, которые считаются верными недостаточно --- мы ничего нового получить не сможем.

\begin{definition}
Правила вывода --- как из утверждений получить новые.

Пусть есть последовательность утверждений: $C_1, C_2, C_3, \cdots$. Для каждого $C_i$ есть три варианта --- быть аксиомой, быть утверждением, в истинность которого мы верим (<<дано>>), или получено при помощи правил вывода, применённых к $C_1, C_2, \cdots, C_{i-1}$. 

\end{definition}

Зачем же нам это нужно? Потому что естественный язык может приводить к заблуждениям.

\begin{example}
	Cheesburger is better than nothing. Nothing is bettter than eternal happiness. Тогда по транзитивности Cheesburger is better than eternal happiness.
\end{example}

\begin{example}
	\begin{enumerate}
		\item{В этой рамочке содержится как минимум одно неверное утверждение. (Представьте, что вокруг этого и следующего утверждений находится рамочка)}
		\item{Среди нас нет долларовых миллиардеров}

	\end{enumerate}
	Если первое утверждение верно, то неверно второе, т.е. миллиардеры есть, если первое неверно, то в рамочке все верные, но противоречие, т.к. первое --- неверное, противоречие.
\end{example}

Эти наглядные примеры показывают необходимость математического языка высказываний, иначе будем попадать в подобные ситуации.

Мы будем рассматривать

\Subsection{Логика высказываний}

Или пропозиционная логика. В ней используются операции $\land, \lor, \to, \lnot$ и операции над множествами.

Нас интересуют тавтологии --- формулы, истинные при любых значениях высказываний.

Какие есть варианты? Можно все тавтологии засунуть в аксиомы и радоваться жизни и ничего больше не делать. Но мы этого не хотим, мы хотим выводить тавтологии из множества аксиом.

У нас будет множество аксиом и единственное правило вывода.

Далее будем считать истинными утверждениями --- тавтологии.

Можно говорить про полноту и корректность логики.

\begin{definition}
	Корректность --- все выводимые утверждения являются тавтологиями (ничего лишнего не выведем)

	Полнота --- имея аксиомы и правила вывода нужно иметь возможность выводить все верные утверждения (ничего не потеряем)
\end{definition}

\begin{definition}
Одиннадцать аксиом исчисления высказываний 
\begin{enumerate}
	\item{$A \to (B \to A)$ --- Из чего угодно следует истина}
	\item{$(A \to (B \to C)) \to ((A \to B) \to (A \to C)) $ --- Дистрибутивность}
	\item{$A \to (A \lor B)$}
	\item{$B \to (A \lor B)$}
	\item{$(A \to C) \to ((B \to C) \to ((A \lor B) \to C))$}
	\item{$(A \land B) \to A$}
	\item{$(A \land B) \to B$}
	\item{$A \to (B \to (A \land B))$}
	\item{$\lnot A \to (A \to B)$ --- Из лжи следует что угодно}
	\item{$(A \to B) \to ((A \to \lnot B) \to \lnot A)$ --- Доказательство от противного }
	\item{$A \lor \lnot A$ --- Аксиома исключённого третьего}
\end{enumerate}
\end{definition}

\begin{definition}
	modus ponens --- единственное правило вывода: $\frac{A, A \to B}{B}$
\end{definition}

\begin{example}
Можем вывести формулу: $(A \lor B) \to (B \lor A)$

$C_1 = A \to (B \lor A)$, аксиома 4. $C_2 = B \to (B \lor A)$, какая-то другая аксиома с переименованием переменных. Далее в пятую аксиому подставим вместо $A$ --- $A$, вместо $B$ --- $(B \lor A)$, вместо $C$ --- $B$, получим $(A \to (B \lor A)) \to ((B \to (B \lor A)) \to ((A \lor B) \to (B \lor A)))$, тогда первая часть это $A'$, а вторая --- $B'$. Но тогда заметим, что левая часть это аксиома, и $A' \to B'$, а значит по правилу вывода получим $B'$. Он, в свою очередь, тоже распадается на два утверждения, из которых первое актиома, и следствие верно (доказали ранее), а значит по правилу вывода последняя скобка $(A \lor B) \to (B \lor A)$ верна, что и хотели доказать.
\end{example}

\begin{theorem}
	Исчисление высказываний корректно
\end{theorem}
\begin{proof}
	\begin{enumerate}
		\item{Все аксиомы --- тавтологии}
		\item{Пусть $C_1, C_2, \cdots, C_k$ --- тавтологии, тогда $C_{k+1}$ --- тавтология: или $C_{k+1}$ --- аксиома, тогда она тавтология, или получена применением modus ponens к $C_1, \cdots, C_k$. У нас есть $C_i$ --- тавтология, и у нас есть $C_i \to C_{k+1}$ --- тавтология. Хотим сказать, что тогда $C_{k+1}$ --- тавтология. Пусть это не так, тогда есть значения, на которых $C_{k+1}$ равно нулю, но тогда при подстановке таких значений в $C_i \to C_{k+1}$ мы получим, что из истины следует ложь.}
	\end{enumerate}
\end{proof}

\begin{lemma}
	Формула $A \to A$ выводима. 
\end{lemma}
\begin{proof}
	Запишем первую аксиому с подстановкой $A = A, B = (A \to A)$, обозначим за $C_1$. $C_2 = A \to (A \to A)$, тоже первая аксиома, но с подстановкой $A=A, B=A$. За $C_3$ возьмём вторую аксиому, с подстановкой $A=A, B=(A \to A), C=A$.

	Последнее будет выглядеть как $(A \to ((A \to A) \to A)) \to (A \to (A \to A)) \to ((A \to A))$, первое как $A \to ((A \to A) \to A)$. Применим modus ponens к $C_1, C_3$, получим $C_4 = (A \to (A \to A)) \to (A \to A)$. Применим modus ponens к $C_2, C_4$. Получим $C_5 = A \to A$. 
\end{proof}

\begin{definition}
	Если формулу можно вывести, то будем обозначать $\vdash$
\end{definition}

\begin{definition}
	$\Gamma \vdash F$, где $F = C_n$, и есть $C_1, C_2, \cdots, C_{n-1}$,  где $C_i$ это или аксиома, или формула из $\Gamma$, или можно вывести через modus ponens.  
\end{definition}
\begin{lemma}
	Лемма о дедукции:

	$\Gamma \vdash A \to B \Leftrightarrow \Gamma, A \vdash B$
\end{lemma}
\begin{proof}
	
	$\Rightarrow$: $\Gamma, A : \cdots, A \to B, A, $ и теперь применим modus ponens к последним и получим $B$, т.е. $\Gamma, A \vdash B$

	$\Leftarrow$: $\Gamma, A \vdash B \Rightarrow \Gamma \vdash A \to B$

	Пусть $C_1, C_2, \cdots, C_k = B$. Допишем везде $A$, получим последовательность $A \to C_1, A \to C_2, \cdots, A \to B$. Хотим показать, что это корректный вывод. Будем делать по индукции, почему можно написать $A \to C+i$: Если $C_i = A$, то получили $A \to A$, что мы уже умеем выводить, если $C_i$ --- аксиома, то допишем $C_i$ т.к. это аксиома, а затем допишем $C_i \to (A \to C_i)$, что аксиома, затем применим modus ponens и получим как раз $(A \to C_i)$, если $C_i \in \Gamma$, то то же самое, что и предыдущий пункт, но теперь $C_i$ имеем право писать, т.к. оно из $\Gamma$. И если $C_i = MP(C_k, C_j), k, j < i$, и $C_j = C_k \to C_i$. Всё остальное до этого мы уже конвертировали в $A \to C_k, A \to C_j$ и хотим вывести $A \to C_i$. Итого есть $A \to C_k, A \to (C_k \to C_i)$, напишем вторую аксиому с подстановкой $A = A, B = C_k, C = C_i$, т.е. $(A \to (C_k \to C_i)) \to ((A \to C_k) \to (A \to C_i))$. Тогда по modus ponens $(A \to C_k) \to (A \to C_i)$, а левая часть уже есть, так что итог --- $A \to C_i$.

	Ура, мы доказали лемму о дедукции.
\end{proof}

Теперь посмотрим на правило $A \to B \to (A \land B)$. По сути это $\varnothing \vdash A \to B \to (A \land B)$ что $\Leftrightarrow A \vdash B \to (A \land B) \Leftrightarrow A, B \vdash A \land B$.

Но зачем же всё это нужно?

Представим, что мы знаем всю теорию групп. В ней три аксиомы: $(ab)c = a(bc), \exists e : ea = ae; \forall a \exists a^{-1}: a a^{-1} = a^{-1} a = e$. Т.е. для доказательства любого факта про группы достаточно знать эти три. 

Хотелось бы в математике тоже сформулировать какое-то число аксиом, и чтобы из них всё следовало. Ну вот примерно в этом направлении мы и двигаемся (но вообще это не очень возможно). 

А вообще мы это изучаем чтобы дальше ориентироваться в более сложных вещей. Пример из аксиом вещественных чисел: $\forall x > 0 \exists y, z : x = y^2 + z^2$. Тут перебрать уже все значения переменных, в отличии от перебора значений булевых переменных, не получится. 

А теперь к булевой логике.

\Subsection{И ещё правила вывода}

Из одиннадцати аксиом и modus ponens можно получить ещё правила вывода, упрощающие жизнь (и домашку):

\begin{enumerate}
	\item{$\Gamma A, B, A \to (B \to (A \land B)), B \to (A \land B), A \land B$, так получили, что $\frac{\Gamma \vdash A\ \ \ \Gamma \vdash B}{\Gamma \vdash A \land B}$}
	\item{$ \Gamma, A \land B \vdash (A \land B) \to A, A \land B, A, (A \land B) \to B, B, \cdots, C$, так получили, что $\frac{\Gamma, A, B \vdash C}{\Gamma, A \land B \vdash C}$}
	\item{$\Gamma, A, B \vdash A \to (B \to (A \land B)), A \land B $, так получили, что $\frac{\Gamma, A \land B \vdash C}{\Gamma, A, B \vdash C}$}
	\item{$\Gamma \vdash A, A \to B, B$, так получили $\frac{\Gamma \vdash A\ \ \ \Gamma, A \vdash B}{\Gamma \vdash B}$. Ну и обобщение есть, когда у нас больше переменных выводится из $\Gamma$}
	\item{$\Gamma \vdash A \to B, A \vdash \lnot B, (A \to B) \to ((A \to \lnot B) \to \lnot A ), \text{Дважды modus ponens: } \lnot A$, так получили $\frac{\Gamma, A \vdash B\ \ \ \Gamma, A \vdash \lnot B}{\Gamma \vdash \lnot A} $}
	\item{$\lnot A \to (A \to B), \lnot A, A, A \to B, B$, Так получили $\Gamma, A, \lnot A \vdash B$}
	\item{$\Gamma, A \lor B \vdash A \to C, B \to C, (A \to C) \to ((B \to C) \to ((A \lor B) \to C)), A \lor B \to C, C$, Так получили $\frac{\Gamma, A \vdash C\ \ \ \Gamma, B \vdash C}{\Gamma, A \lor B \vdash C}$}
	\item{Подставим в 7 $A$ и $\lnot A$, получим $\frac{\Gamma, A \vdash C\ \ \ , |Gamma, \lnot A \vdash C}{}Gamma, A \lor \lnot A \vdash C$, а аксиому можно из итогов выкинуть и получить $\frac{\Gamma, A \vdash C\ \ \Gamma, \lnot A \vdash C}{\Gamma \vdash C}$}
\end{enumerate}

\Subsection{И обратно к утверждениям}

\begin{statement}
    $\vdash \lnot \lnot A \to A$
\end{statement}
\begin{proof}
    $\lnot \lnot A \vdash A$

    $A \lor \lnot A, \lnot \lnot A \vdash A$

    Подставим $A$ и $\lnot A$ к $\lnot \lnot A$, получим

    $A, \lnot \lnot A \vdash A$ поскольку оно есть среди посылок, $\lnot A, \lnot \lnot A \vdash A$ по правилу вывода из противоречия (шестое).  А теперь по седьмому объединили полученные два утверждения.
\end{proof}

\begin{lemma}
	$A, B \vdash A \land B, A \lor B, A \to B$

	$A, \lnot B \vdash \lnot A \land B, A \lor B, \lnot (A \to B)$

	$\lnot A, B \vdash \lnot(A \land B), A \lor B, A \to B$

	$\lnot A, \lnot B \vdash \lnot(A \land B), \lnot(A \lor B), A \to B$

	$\lnot A \vdash \lnot A$

	$A \vdash \lnot(\lnot A)$

	В доказательство предыдущего: $A, \lnot A \vdash A; A, \lnot A \vdash \lnot A$

	Дванадцать первых утверждений доказываются или через modus ponens, или через правило вывода номер пять (которым можно отрицания получать, выводя противоречие). Конкретные доказательства, увы, упущены.
\end{lemma}

А вообще её можно было несколько иначе сформулировать: давайте писать $A^x = \left\{\begin{aligned} A &, x = 1\\ \lnot A &, x = 0  \end{aligned} \right.$

\begin{lemma}
	$\varphi(A_1, A_2, \cdots, A_k)$ --- формула, $f(x_1, x_2, \cdots, x_k)$ --- функция, вычисляющая эту формулу

	$A_1^{x_1}, A_2^{x_2}, \cdots, A_k^{x_k} \vdash \varphi(A_1, \cdots, A_k)^{f(x_1, \cdots x_k)}$

	Пример:


	Тогда, подставив туда $\lor$ и единицы  получим $A_1^1, A_2^1 \vdash (A_1 \lor A_2)^{1 \lor 1} = a_1 \lor A_2$. Аналогично можно подставлять нули и единицы в произвольных комбинациях с соответсвующими изменениями результата.
\end{lemma}
\begin{proof}
	Доказывать будем по индукции по сложности построения формулы $\varphi$. База: если $\varphi(A_1, \cdots, A_k) = A_i$, то $A_1^{x_1}, \cdots, A_k^{x_k} \vdash A_i^{x_i}$

	Переход: $\varphi = \varphi_1 (op) \varphi_2$, где $\varphi$ соответствует $f$, $\varphi_1$ соответствует $g$, $\varphi_2$ соответствует $h$, а $(op)$  соответствует $\land, \lor, \to, \cdots$. Тогда $f(x_1, \cdots, x_k) = g(x_1, \cdots, x_k) (op) h(x_1, \cdots, x_k)$. Знаем (по ИП), что $A_1^{x_1}, \cdots, A_k^{x_k} \vdash \varphi_1(A_1, \cdots, A_k)^{g(x_1, \cdots x_k)}, \varphi_2(A_1, \cdots, A_k)^{h(x_1, \cdots, x_k)}$. 

	Ну и дальше тут происходило что-тоо особо странное, если честно. Кто помнит и понял --- допишите этот момент, пожалуйста.
\end{proof}

\begin{theorem}
	Любая тавтология выводима
\end{theorem}
\begin{proof}
	Возьмём произвольную тавтологию $\varphi$. Пускай он зависит от $A_1, A_2, \cdots, A_k$. Умеем выводить $A_1, A_2, \cdots, A_k \vdash \varphi(A_1, \cdots, A_k)^{\varphi(1, \cdots, 1)}; \lnot A_1, A_2, \cdots, A_n \vdash \varphi(A_1, \cdots, A_k)^{\varphi(0, 1, \cdots, 1)}; \cdots, \lnot A_1, \lnot A_2, \cdots, \lnot A_k \vdash \varphi(A_1, \cdots, A_k)$. Но мы знаем, что $\varphi$ тавтология, поэтому верхние $\varphi(1/0, 1/0, \cdots, 1/0)$ можно убрать. Заметим, что $A_1, \cdots, A_k \vdash \varphi, \lnot A_1, \cdots, A_k \vdash \varphi$, а значит $A_1 \lor \lnot A_1, A_2, \cdots, A_k \vdash \varphi$, а значит можно убрать $A_1 \lor \lnot A_1$. Останемся с хвостом $A_2, \cdots, A_k \vdash \varphi$, потом к $A_2$ можно применить похожий трюк, и в итоге останемся с тем, что $\vdash \varphi$, т.е. $\varphi$ выводится из ИВ. 
\end{proof}

А теперь докажем ещё иным способом:

Но сначала введём определения:

\begin{definition}
	Множество (/набор) формул $\Gamma$ --- совместно, если $\exists$ набор значений, на котором все формулы из $\Gamma$ вычисляются в единицу.
\end{definition}

\begin{definition}
	$\Gamma$ является противоречивым множеством формул, если $\Gamma \vdash F$ и $\Gamma \vdash \lnot F$. Если же $\Gamma$ не противоречиво, то оно непротиворечиво. 
\end{definition}

\begin{theorem}
	Множество непротиворечиво тогда и только тогда когда оно совместно. 
\end{theorem}

Перед доказательством поймём, почему из него следует полнота и корректность. 

\begin{theorem}
	Из этого следует теорема о корректности --- всё выводимое является тавтологиями
\end{theorem}
\begin{proof}
	Пусть $\vdash \varphi$, тогда $\lnot \varphi$ противоречиво, т.к. $\lnot \varphi \vdash \varphi$ и $\lnot \varphi \vdash \lnot \varphi$. Раз противоречиво, то несовместно. Значит нет набора, на котором оно есть единица, тогда $\varphi$ от всех наборов равняется единице. Корректность доказана. 
\end{proof}

\begin{theorem}
	Теорема о полносте --- любая тавтология выводится
\end{theorem}
\begin{proof}
	$\varphi$ --- тавтология, тогда $\lnot \varphi$ --- несовместно, а тогда оно противоречиво (всё по той же недоказанной теореме). А раз противоречиво. Тогда можно $\vdash \lnot \lnot \varphi$, т.к. $\lnot \varphi \vdash G, \lnot \varphi \vdash \lnot G$. А тогда $\vdash \lnot \lnot \varphi, \lnot \lnot \varphi \to \varphi$
\end{proof}

\begin{statement}
	$\Gamma$ --- совместно, тогда $\Gamma$ непротиворечиво. Пусть $F_1, F_2, \cdots, F_k$ (формул, возможно, бесконечность) $\vdash$. Ну а потом там modus ponens бахаем и пытаемся радоваться жизни, я снова проиграл, кто понял + воспринял --- распишите пж.
\end{statement}
\begin{statement}
	$\Gamma$ --- непротиворечиво, тогда $\Gamma$ --- совместно. 
\end{statement}
\begin{proof}
	Ну там если есть прямые формулы вида $x_1, x_2, \lnot x_3$ то для них значения очевидны. И тогда если есть выполняющий набор, то это такой.

	А теперь хотим $\Gamma$ вложить в непротиворечивый $\Gamma'$, содержащий $x_1$ или $\lnot x_1$. И так последовательно продобавляем для каждой переменной $x_i$ или $\lnot x_i$. 
\end{proof}
\begin{definition}
	$\Gamma$ --- полное, если $\forall \varphi \Gamma \vdash \varphi$ или $\Gamma \vdash \lnot \varphi$.
\end{definition}
\begin{statement}
	$\forall \Gamma $ непротиворечивых $\exists$ полное $\Gamma': \Gamma \subset \Gamma'$ и $\Gamma'$ --- непротиворечиво.

	Пусть формул счётное число, тогда выпишем их по порядку. И пойдём по ним в этом порядке. $\Gamma_{i} = \left\{ \begin{aligned} \Gamma_{i-1} &\cup \varphi_i \\ \Gamma_{i-1} &\cup \lnot \varphi_i \end{aligned} \right.$ что $\Gamma_{i}$ --- непротиворечиво. Хотим получить это, зная, что $\Gamma_{i-1}$ --- непротиворечиво. Пусть $\Gamma_{i-1} \vdash \varphi$, тогда $\Gamma_i = \Gamma_{i-1} \cup \varphi_i$. А если $\Gamma \vdash \lnot \varphi_i$, то $\Gamma_i = \Gamma_{i-1} \cup \lnot \varphi_i$. А если же ни одно не выводится, то возьмём произвольное, в частности $\Gamma_i = \Gamma-{i-1} \cup \varphi_i$. 

	Почему не возникло противоречий? Пусть возникли, тоогда $\Gamma_{i-1}, \varphi_i \vdash G; \text{ и } \vdash \lnot G$, но тогда просто $\Gamma_{i-1} \vdash \lnot \varphi_i$. А мы в третьем случае предположили, что ни то, ни это выводить не умеем, противоречие. 
\end{statement}

\begin{proof}
	Продолжение предыдущего доказательства: теперь умеем выписывать ещё и все переменные, почему такое означение будет корректным? Узнаем на следующей паре
\end{proof}

Вспомните определение совместного множества формул. И определение противоречивого. 

\begin{proof}
	Хотим доказать, что если $\Gamma$ не противоречиво, то $\Gamma$ совместно.

	Уже щнаем, что можем дополнить $\Gamma$ до полного. Тогда $\Gamma' \vdash x_1$ или $\Gamma' \vdash \lnot x_1$, а значит можно означить все переменные. Осталось понять, почему все формулы будут выполнены.

	Пусть есть $\varphi(x_{i_1}, \cdots x_{i_k}) = 0$. У нас была лемма, что $A_1^{x_1}, \cdots A_k^{x_k} \vdash \varphi(A_1, \cdots,  A_k)$. Тогда мы можем $\Gamma' \vdash \lnot \varphi$. Но при этом $\Gamma' \vdash \varphi$ т.к. $\varphi \in \Gamma'$. Противоречие. Значит были противоречивы. Противоречие.
\end{proof}

Теперь знаем, что $\Gamma$ не противоречиво $\Leftrightarrow$ $\Gamma$ совместно

\begin{theorem} О компактности

	$\Gamma$ совместно $\Leftrightarrow$ совместно любое конечное подмножество формул из $\Gamma$
	
\end{theorem}
\begin{proof}
	Если $\Gamma$ совместно, то очевидно и любое конечное подмножество будет совместным.

	В обратную сторону от противного. Пусть $\Gamma$ не совместно. Тогда, по предыдущей теореме, $\Gamma$ противоречиво. Но тогда $\exists C_1, C_2, \cdots C_k = \varphi$ и $C_1', C_2', \cdots, C_k' = \lnot \varphi$
	Это конечное множество формул (объединение этих двух цепочек). Оно использует только аксиомы и конечное число формул из $\Gamma$. А тогда мы получили конечное число формул из $\Gamma$, из которых выводится противоречие. Вот и конечное несовместное множество.
\end{proof}

А теперь задача --- хотим покрасить плоскость в минимальное число цветов так, чтобы любые две точки на расстоянии один были покрашены в разные цвета. Какое минимальное количество цветов? Понятно, что в девять наверно можно, а в два точно нельзя (равносторонний треугольник со стороной один). Оценку снизу можно предъявить используя конечное число точек. 

Вопрос: верно ли, что достаточно только конечных наборов? Точнее, верно ли, что если плоскость можно покрасить не меньше, чем в $k$ цветов, то для $k-1$ цветов будет существовать пример на конечном числе точек?

\begin{definition}
	Обобщённый граф --- подмножество декартового квадрата вершин. Но вершин теперь может быть бесконечность.
\end{definition}
\begin{definition}
	Граф двудольный $\Leftrightarrow$ в нём нет нечётных циклов. 
\end{definition}

\begin{theorem}
Верно ли это для обобщённых графов? 
\end{theorem}
\begin{proof}
Давайте каждой вершине в нашем обобщённом графе присвоим $x_i$, где индексируем не обязательно натуральными числами. Если мы означим ей значение 0, то она будет лежать в левой доле, а если 1, то в правой. Тогда двудольность --- означить все и проверить, что все рёбра $(x \lor y) \land (\lnot x \lor \lnot y)$. Так мы для каждого ребра создали формулу. Получили набор формул $\Gamma$.

По теореме о компактности, $\Gamma$ совместно, если для любое конечное подмножетство совместно. А когда оно может быть несовместно? Когда есть конечный цикл нечётной длины.
\end{proof}

А теперь вернёмся к конечным наборам. Для каждой точки создадим набор переменных. Так для точки $(i, j)$ получили набор $x_1^{i, j}, x_2^{i, j}, \cdots, x_k^{i, j}$. Так хотим, чтобы $x_1 = \cdots x_{p-1} = x_{p+1} = \cdots x_k = 0$ и $x_p = 1$. Т.е. $\bigcup_{p=1} (x_p \land \lnot x_1 \land \lnot x_2 \land \cdots \land \lnot x_{p-1} \land \lnot x_{p+1} \land \cdots \land \lnot x_k)$. Теперь добавим условия на то, что две вершины не могут быть покрашены в один цвет: $\forall p \in [1, k] \lnot x_p^{i, j} \lor \lnot x_p^{i', j'}$. И снова нам нужно совместное множество формул а значит если оно несовместно, то найдётся несовместное конечное подмножество, по которому набирается контрпример из конечного числа точек.

Шаг в сторону: ШОК КОНТЕНТ: Десятая аксиома выводится из аксиом 1-9 и 11.

Сама аксиома: $(A \to B) \to ((A \to \lnot B) \to \lnot A)$. Заметим, что она не используется для доказательства леммы о Дедукции. Тогда нам нужно доказать, что $A \to B, A \to \lnot B \vdash \lnot A$. А ещё у нас было правило разбора случаев. А тогда $(A \to C) \to ((B \to C) \to (A \lor B) \to C)$, и сидим с цепочкой из modus ponens $A \to B, A, B, A \to \lnot B, \lnot B, B \to (\lnot B \to \lnot A)$. 

А вот аксиому 11 ($A \lor \lnot A$) выкинуть нельзя. Но почему?

А давайте введём \textbf{мегалогику}. В ней ещё будет, помимо истины и лжи, мистическое \textbf{наверное}. 

А дальше можно увидеть, как русские буквы не отображаются в формулах :)

Сидим с $И, Л, Н$. Хотим доопределить всё так, чтобы старое не сломалось. Хотим доопределить таблицы истинности для случаев, когда они из операндов --- наверно.

Тавтологии --- те формулы, куда подстановка любых значений из тройки результирует в истину. 

Сопоставим $И = 1, Л = 0, Н = \frac12$. $x \land y = \min(x, y)$. $x \lor y = \max(x, y)$. А ещё $\lnot Н = Л, \lnot(\not Н) = И$. $И \to x = x, Л \to x = И, (Н \to Л) = Л, (Н \to Н) = И, (Н \to И) = И$.

Аксиомы с первой по десятую сохраним, они остаются тавтологиями. Но вот одиннадцатая аксиома перестала быть тавтологией: $(Н \lor \lnot Н) = (Н \lor Л) = Н$

А теперь предположим, что из первых десяти аксиом выводится одиннадцатая. Тогда пусть есть $C_1, C_2, \cdots, C_k = A \lor \lnot A$, где $C_i$ --- аксиома или применение modus ponens. Тогда пусть $A = Н$. Мы знаем, что при modus ponens $(C_k \to C_j)$ истинно. Тавтологии у нас были там все как и в 2-системе, так и в 3-системе, но итоговый вывод не является тавтологиев в 3-системе, противоречие/проблема. 

Логика, в которой аксиомы с 1 по 10, называется интуиционистской логикой.

Обратно к компактности. Выполняется для любого числа переменных, конечного, счётного, несчётного...

Пусть есть переменные $x_1, \cdots, x_n$. Тут было что-то про выполнимость, классы и $2^{2^n}$ и про выполнимость класса. Вставить текст.

Пускай переменные $x_1, \cdots, x_n, \cdots$. Теперь рассмотрим какие-то наборы значений этих переменных.  Различных наборов значений переменных больше, чем континум ($2^{2^{\N}}$). Пусть $\mathcal{A}$ --- набор означиваний. Хотим построить множество формул $\Gamma$ так, чтобы значение лежало в наборе тогда и только тогда, когда любая формула из $\Gamma$ вычисляется в истину. Если хотим один набор, то его легко задать, явно описав значения через $x_1, \lnot x_2, \cdots$. Но подмножеств формул --- континум, а означиваний --- больше, чем континум. А тогда существует какой-то набор означиваний, который нельзя задать формулами. Печально!


