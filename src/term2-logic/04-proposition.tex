Просто аксиом и утверждений, которые считаются верными недостаточно --- мы ничего нового получить не сможем.

\begin{definition}
Правила вывода --- как из утверждений получить новые.

Пусть есть последовательность утверждений: $C_1, C_2, C_3, \cdots$. Для каждого $C_i$ есть три варианта --- быть аксиомой, быть утверждением, в истинность которого мы верим (<<дано>>), или получено при помощи правил вывода, применённых к $C_1, C_2, \cdots, C_{i-1}$. 

\end{definition}

Зачем же нам это нужно? Потому что естественный язык может приводить к заблуждениям.

\begin{example}
	Cheesburger is better than nothing. Nothing is bettter than eternal happiness. Тогда по транзитивности Cheesburger is better than eternal happiness.
\end{example}

\begin{example}
	\begin{enumerate}
		\item{В этой рамочке содержится как минимум одно неверное утверждение. (Представьте, что вокруг этого и следующего утверждений находится рамочка)}
		\item{Среди нас нет долларовых миллиардеров}

	\end{enumerate}
	Если первое утверждение верно, то неверно второе, т.е. миллиардеры есть, если первое неверно, то в рамочке все верные, но противоречие, т.к. первое --- неверное, противоречие.
\end{example}

Эти наглядные примеры показывают необходимость математического языка высказываний, иначе будем попадать в подобные ситуации.

Мы будем рассматривать

\Subsection{Логика высказываний}

Или пропозиционная логика. В ней используются операции $\land, \lor, \to, \lnot$ и операции над множествами.

Нас интересуют тавтологии --- формулы, истинные при любых значениях высказываний.

Какие есть варианты? Можно все тавтологии засунуть в аксиомы и радоваться жизни и ничего больше не делать. Но мы этого не хотим, мы хотим выводить тавтологии из множества аксиом.

У нас будет множество аксиом и единственное правило вывода.

Далее будем считать истинными утверждениями --- тавтологии.

Можно говорить про полноту и корректность логики.

\begin{definition}
	Корректность --- все выводимые утверждения являются тавтологиями (ничего лишнего не выведем)

	Полнота --- имея аксиомы и правила вывода нужно иметь возможность выводить все верные утверждения (ничего не потеряем)
\end{definition}

\begin{definition}
Одиннадцать аксиом исчисления высказываний 
\begin{enumerate}
	\item{$A \to (B \to A)$ --- Из чего угодно следует истина}
	\item{$(A \to (B \to C)) \to ((A \to B) \to (A \to C)) $ --- Дистрибутивность}
	\item{$A \to (A \lor B)$}
	\item{$B \to (A \lor B)$}
	\item{$(A \to C) \to ((B \to C) \to ((A \lor B) \to C))$}
	\item{$(A \land B) \to A$}
	\item{$(A \land B) \to B$}
	\item{$A \to (B \to (A \land B))$}
	\item{$\lnot A \to (A \to B)$ --- Из лжи следует что угодно}
	\item{$(A \to B) \to ((A \to \lnot B) \to \lnot A)$ --- Доказательство от противного }
	\item{$A \lor \lnot A$ --- Аксиома исключённого третьего}
\end{enumerate}
\end{definition}

\begin{definition}
	modus ponens --- единственное правило вывода: $\frac{A, A \to B}{B}$
\end{definition}

\begin{example}
Можем вывести формулу: $(A \lor B) \to (B \lor A)$

$C_1 = A \to (B \lor A)$, аксиома 4. $C_2 = B \to (B \lor A)$, какая-то другая аксиома с переименованием переменных. Далее в пятую аксиому подставим вместо $A$ --- $A$, вместо $B$ --- $(B \lor A)$, вместо $C$ --- $B$, получим $(A \to (B \lor A)) \to ((B \to (B \lor A)) \to ((A \lor B) \to (B \lor A)))$, тогда первая часть это $A'$, а вторая --- $B'$. Но тогда заметим, что левая часть это аксиома, и $A' \to B'$, а значит по правилу вывода получим $B'$. Он, в свою очередь, тоже распадается на два утверждения, из которых первое актиома, и следствие верно (доказали ранее), а значит по правилу вывода последняя скобка $(A \lor B) \to (B \lor A)$ верна, что и хотели доказать.
\end{example}

\begin{theorem}
	Исчисление высказываний корректно
\end{theorem}
\begin{proof}
	\begin{enumerate}
		\item{Все аксиомы --- тавтологии}
		\item{Пусть $C_1, C_2, \cdots, C_k$ --- тавтологии, тогда $C_{k+1}$ --- тавтология: или $C_{k+1}$ --- аксиома, тогда она тавтология, или получена применением modus ponens к $C_1, \cdots, C_k$. У нас есть $C_i$ --- тавтология, и у нас есть $C_i \to C_{k+1}$ --- тавтология. Хотим сказать, что тогда $C_{k+1}$ --- тавтология. Пусть это не так, тогда есть значения, на которых $C_{k+1}$ равно нулю, но тогда при подстановке таких значений в $C_i \to C_{k+1}$ мы получим, что из истины следует ложь.}
	\end{enumerate}
\end{proof}

\begin{lemma}
	Формула $A \to A$ выводима. 
\end{lemma}
\begin{proof}
	Запишем первую аксиому с подстановкой $A = A, B = (A \to A)$, обозначим за $C_1$. $C_2 = A \to (A \to A)$, тоже первая аксиома, но с подстановкой $A=A, B=A$. За $C_3$ возьмём вторую аксиому, с подстановкой $A=A, B=(A \to A), C=A$.

	Последнее будет выглядеть как $(A \to ((A \to A) \to A)) \to (A \to (A \to A)) \to ((A \to A))$, первое как $A \to ((A \to A) \to A)$. Применим modus ponens к $C_1, C_3$, получим $C_4 = (A \to (A \to A)) \to (A \to A)$. Применим modus ponens к $C_2, C_4$. Получим $C_5 = A \to A$. 
\end{proof}

\begin{definition}
	Если формулу можно вывести, то будем обозначать $\vdash$
\end{definition}

\begin{definition}
	$\Gamma \vdash F$, где $F = C_n$, и есть $C_1, C_2, \cdots, C_{n-1}$,  где $C_i$ это или аксиома, или формула из $\Gamma$, или можно вывести через modus ponens.  
\end{definition}
\begin{lemma}
	Лемма о дедукции:

	$\Gamma \vdash A \to B \Leftrightarrow \Gamma, A \vdash B$
\end{lemma}
\begin{proof}
	
	$\Rightarrow$: $\Gamma, A : \cdots, A \to B, A, $ и теперь применим modus ponens к последним и получим $B$, т.е. $\Gamma, A \vdash B$

	$\Leftarrow$: $\Gamma, A \vdash B \Rightarrow \Gamma \vdash A \to B$

	Пусть $C_1, C_2, \cdots, C_k = B$. Допишем везде $A$, получим последовательность $A \to C_1, A \to C_2, \cdots, A \to B$. Хотим показать, что это корректный вывод. Будем делать по индукции, почему можно написать $A \to C+i$: Если $C_i = A$, то получили $A \to A$, что мы уже умеем выводить, если $C_i$ --- аксиома, то допишем $C_i$ т.к. это аксиома, а затем допишем $C_i \to (A \to C_i)$, что аксиома, затем применим modus ponens и получим как раз $(A \to C_i)$, если $C_i \in \Gamma$, то то же самое, что и предыдущий пункт, но теперь $C_i$ имеем право писать, т.к. оно из $\Gamma$. И если $C_i = MP(C_k, C_j), k, j < i$, и $C_j = C_k \to C_i$. Всё остальное до этого мы уже конвертировали в $A \to C_k, A \to C_j$ и хотим вывести $A \to C_i$. Итого есть $A \to C_k, A \to (C_k \to C_i)$, напишем вторую аксиому с подстановкой $A = A, B = C_k, C = C_i$, т.е. $(A \to (C_k \to C_i)) \to ((A \to C_k) \to (A \to C_i))$. Тогда по modus ponens $(A \to C_k) \to (A \to C_i)$, а левая часть уже есть, так что итог --- $A \to C_i$.

	Ура, мы доказали лемму о дедукции.
\end{proof}

Теперь посмотрим на правило $A \to B \to (A \land B)$. По сути это $\varnothing \vdash A \to B \to (A \land B)$ что $\Leftrightarrow A \vdash B \to (A \land B) \Leftrightarrow A, B \vdash A \land B$.

