\Subsection{Равномощные множества}
...тут я ещё не начинал записывать
\begin{definition}
    Множества $A$ и $B$ равномощны, если $\exists$ биекция между $A$ и $B$ 
\end{definition}
\begin{remark}
	Равномощность является отношением эквивалентности. Очев.
\end{remark}
Возможная, но не совсем корректная трактовка --- равномощные множества --- содержащие равное количество элементов. Для конечных это действительно верно.

\begin{definition}
	Множество называется счётным, если оно равномощно множеству натуральных чисел.
\end{definition}

Пояснение: мы считаем натуральными числами множество $\{1, 2, 3, \cdots\}$, но никакой разницы с $\{0, 1, 2, 3, \cdots\}$ нет, т.к. существует биекция из одного в другое --- $i \to i - 1$.

\begin{example}
	Чётные числа --- счётное множество. Биекция --- $i \to 2 \times i$
\end{example}

\begin{lemma}
$A, B$ --- счётны $\Rightarrow$ $A \cup B$ --- счётно при $A \cap B = \emptyset$.
\end{lemma}
\begin{proof}
	$\exists f: N \to A \Rightarrow$ $A : \{a_1, a_2, a_3, \cdots\}$

	Аналогично $B: \{b_1, b_2, b_3, \cdots\}$

	Тогда запишем:

	$C: \{a_1, b_1, a_2, b_2, \cdots\}$, $c_{2i - 1} = a_i, c_{2i} = b_i$
\end{proof}
\begin{consequence}
$\mathbb{Z}$ --- счётно.
\end{consequence}
\begin{proof}
	$\mathbb{Z} = \{0, 1, 2, \cdots\} \cup \{-1, -2, -3, \cdots\}$, первое равномощно $\mathbb{N}$, как и второе, а значит $\mathbb{Z}$ --- счётно по предыдущей лемме.
\end{proof}
\begin{lemma}
	$B$ --- счётное, $A \subset B$, тогда $A$ --- конечное или счётное.
\end{lemma}
\begin{proof}
	$B$ ---  счётно, тогда можно записать $B: \{b_1, b_2, \cdots\}$

	Раз $A$ подмножество, то просто часть элементов отсутствует. Тогда мы пойдём сопоставлять числа оставшимся элементам. Первое оставшееся --- 1, второе --- 2, и т.д. Тогда или в какой-то момент у нас закончатся оставшиеся числа, т.е. найдётся то, после которого нет оставшихся, и тогда $A$ --- конечно, или мы построим биекцию между $A$ и натуральными числами. Это биекция, поскольку это инъекция и сюръекция (мы каждому натуральному поставили число, и всем элементам $A$ что-то одно сопоставили)
\end{proof}

\begin{theorem}[Лемма]
Любое бесконечное множество содержит счётное подмножество.
\end{theorem}
\begin{definition}
	Множество $X$ бесконечно, если $\forall i \in \mathbb{N}$ можно найти $i$ различных элементов из $X$.
\end{definition}
\begin{proof}
	Возьмём элемент из $X$. Назовём его $a_1$. Если в $X$ не осталось элементов, значит в нём был всего один элемент. Иначе возьмём из $X$ какой-то другой элемент, назовём его $a_2$. Если снова не осталось, то было всего два элемента. И так далее, построили $Y = \{a_1, a_2, a_3, \cdots\}$, и $Y \subset X, f: \mathbb{N} \Leftrightarrow Y$.
\end{proof}
\begin{example}
	Счётное множество, для которого мы таким процессом не докажем счётность: $X = \{1, 2, 3, \cdots\}$, $Y = \{2, 4, 6, \cdots\}$. Иными словами, мы все элементы из $X$ далеко не обязательно вытащим.
\end{example}
\begin{example}
	Множество $(0, 1)$ не является счётным.
\end{example}
\begin{consequence}
$A_1, A_2, A_3, \cdots, A_k$ --- счётны $\therefore$ $A_1 \cup A_2 \cup \cdots A_k$ --- счётно
\end{consequence}
\begin{proof}
    Для дизъюнкных всё хорошо понятно. Для недизъюнктных:

    Посмотрим на $A_1$ и $A_2 \\ A_1$. Оба счётны, а тогда $A_1 \cup A_2 = A_1 \sqcup (A_2 \\ A_1)$

    Теперь воспользуемся индукцией по $k$:
    База: $A_1, A_2$ счётны по условию, тогда $A_1 \cup A_2$ тоже счётно.
    Тогда $(A_1 \cup A_2 \cup A_3 \cdots A_k) = ((\cdots((A_1 \cup A_2) \cup A_3) \cup \cdots) \cup A_k)$
\end{proof}
\begin{lemma}
$A_1, A_2, \cdots$ --- счётное число счётных множеств, т.е. для любого $i \exists A_i$.

Тогда $A_1 \cup A_2 \cup \cdots$ тоже счётно.
\end{lemma}
\begin{proof}
	$A_1$ счётно, тогда $A_1: \{a_{11}, a_{12}, a_{13}, \cdots\}$. 
	Аналогично $A_2: \{a_{21}, a_{22}, a_{23}, \cdots\}$
	И так далее ещё счётное число строк.

	Теперь нам нужно эту таблицу представить в виде последовательности.
	Будем ходить по диагоналям: $a_{11}, a_{12}, a_{21}, a_{13}, a_{22}, a_{31}, \cdots$

	Утверждение --- любой элемент будет выписан. Рассмотрим элемент множества $i$ номер $j$, тогда оно будет на $i+j$-ой диагонали, а значит его номер точно не будет превышать $(i+j)^2$. Тогда получаем, что любой элемент будет выписан.

	Это всё для непересекающихся множеств, а для пересекающихся --- давайте просто не выписывать элементы, которые уже выписали. 
\end{proof}
\begin{exerc}
	$\mathbb{N}, \mathbb{Z}$ --- счётны. $[0, 1)$ --- несчётно (просто знаем), знаем $\mathbb{R}$ --- несчётно. $\mathbb{Q}$ --- счётно или нет?
\end{exerc}

\begin{proof}
	$\mathbb{Q}_+$ счётно. Давайте представим его в виде $A_1 \cup A_2 \cup \cdots$, где $A_i = \{\frac{m}i | m \in \mathbb{N}\}$. Т.к. любое из $\mathbb{Q}_+$ так представляется, то в такое объединение попадёт всё $\mathbb{Q}_+$.
\end{proof}
\begin{lemma}
	$A, B$ --- счётны, тогда $A \times B$ --- счётно.
\end{lemma}
\begin{proof}
	$A, B$ --- счётны, тогда $A: \{a_1, a_2, \cdots\}$, $B: \{b_1, b_2, \cdots\}$
	Элементы из $A \times B$ выглядят так: $(a_i, b_j)$, тогда давайте запишем следующее:

	$A_1 = a_1 \times B = \{(a_1, b_1), (a_1, b_2), (a_1, b_3), \cdots\}$, $A_2$ аналогично, и так далее.
	Тогда каждое $A_i$ --- счётно, и их счётное число, значит их объединение, которое и есть $A \times B$ счётно, по доказанной лемме.
\end{proof}

Двигаемся к несчётным множествам.

\begin{lemma}
Пусть $A$ бесконечно, а $B$ --- конечно. Тогда $A \cup B$ равномощно $A$.
\end{lemma}
\begin{proof}
	$B$ заменим на $B' = B / A$. $B'$ или станет пустым, или останется конечным.

	Очевидно, что $A \cup B = A \sqcup B'$

	У $A$ есть счётное подмножество $A' = \{a_1, a_2, \cdots\}$, тогда $A = (A\setminus A') \cup A'$.

	Хотим построить биекцию между $A = A' \cup (A\setminus A')$ и $A' \cup B' \cup (A\setminus A')$

	Между частями $(A \setminus A')$ построим тождественную биекцию. А $a_i$ будем отображать в $b_i$, если $i \le k$, а в $a_{i-k}$ если $i > k$. Понятно, что это биекция. Все элементы возьмём как из $B$, так и из $a_i$.
\end{proof}

\begin{remark}
Доказательство можно модифицировать для случая, когда $B$ счётно. Тогда давайте на последнем шаге чётные отображать в $a_i$, а нечётные --- в $b_i$. 
\end{remark}
\begin{theorem}
	Множество $X$ последовательностей (бесконечных) из нулей и единиц не счётно. (Бинарные строки бесконечной длины)
\end{theorem}
\begin{proof}
	От противного: пусть счётен, тогда есть биекция $f \colon \mathbb{N} \to X$. Тогда выпишем последовательности $f(1), f(2), f(3), \cdots$. 
	А теперь воспользуемся диагональным (методом Кантора): Посмотрим на элемент $a_{11}$, возьмём элемент $1 - a_{11}$. Затем на элемент $a_{22}$, возьмём $1 - a_{22}$. И так далее, строим последовательность $1 - a_{ii}$. Получили бесконечную последовательность нулей и единиц, значит она элемент $X$. Но при этом она не может быть любой $i$-ой последовательностью, поскольку её $i$-ый элемент не совпадает с $i$-ым элементом строки $i$ по тому, как мы строили нашу последовательность.
	Противоречие. Значит мы не можем вот так вот выписать наши элементы $X$, значит биекции $f$ не существует. 
\end{proof}
\begin{consequence}
	Множество чисел из отрезка $[0, 1]$ несчётно.
\end{consequence}
\begin{proof}
	Покажем равномощность множеству $X$ из теоремы. Из бесконечной последовательности число получить легко --- припишем слева <<$0,$>>, а все элементы последовательности запишем слитно. Могло показаться, что получили биекцию, но нет. У нас разные последовательности могут соответствовать одному числу --- $0,100000000\cdots$ и $0,011111111\cdots$  --- разные последовательности, но являются одним числом. 
	Возьмём две последовательности --- $0,a_{11}a_{12}a_{13}\cdots$, $0,a_{21}a_{22}a_{23}\cdots$. Утверждение --- они задают одно число тогда и только тогда, когда они имеют один префиксы, а затем у одного числа идёт единица и после только нули, а у второго ноль и затем только единицы.
	Идём слева направо и найдём первый момент, когда они отличаются. В одном ноль, во втором единица. Далее всё идёт сколько0то, как мы предсказали, затем, что-то разойдётся и там можно оценить, что числа у нас уже отличаются на что-то, что не сможем покрыть дальнейшим. Спасибо Близнецу за успешно закрытую собой доску... Но там в любом случае очев $=D$
	А все числа такого вида это просто $\mathbb{Q}$ (или что-то такого рода)((На самом деле оно даже не $\mathbb{Q}$, там только дроби вида сумма какого-то конечного числа отрицательных степеней двойки, что есть подмножество $\mathbb{Q}$)).
	Тогда $X \equiv [0, 1] \sqcup (\mathbb{Q} \cap [0, 1])$. Результат пересечения счётен, а значит объединение равномощно бесконечной левой части, т.е. $X \equiv [0, 1]$
\end{proof}
Теперь знаем, что натуральные счётны, чётные счётны, целые счётны, рациональные положительные счётны, просто рациональные счётны. А вот действительные уже несчётны, т.к. содержат $[0, 1]$.

\begin{example}
	Множество точек границ треугольника и вписанного круга равномощны, т.к. можно построить биекцию из центра. 
\end{example}

ТУТ ПРОПУЩЕНА ЛЕКЦИЯ. ДОСАДНО.

\Subsection{Об операциях над мощностями}

Если хотим сложить множества (мощности), то нам нужна мощность следующего множества

{
\large
$$
A \times \{0\} \cup B \times \{1\}
$$
}

О корректности --- если выбирать разные множества одной мощности, то можно построить биекцию и не париться.

Очевидным образом коммутативны.

Произведение мощностей, ожидаемо, мощность произведения множеств.

С возведением в степень чуть сложнее: пусть $|A| = a, |B| = b$, то $a^b = |A^B|$, где последнее --- множество всех функций, действующих из $B$ в $A$.

Хотим проверить, что $A^{B \sqcup C} = A^B \times A^C$. Имеем $g : B \to A, h : C \to A$, и функция $f : B \cup C \to A$ взаимооднозначно определяет $g $ и $h$.

Теперь хотим проверить, что $(ab)^c = a^c \times b^c$. Слева имеем $\{f : C \to A \times B\}$, а справа $\{ f : C \to A  \} \times \{g : C \to B\}$. Но тогда заметим, что там условно у первых функций есть две координаты, мощно рассмотреть проекции  на $A$ и на $B$ и всё будет ок.

Остаётся $(a ^ b)^c = a^{b \times c}$. По сути $a^{b \times c}$ это $\{f | f : B \times C \to A\}$. Что плюс-минус есть $f_c(x) = f(x, c)$ --- как только мы фиксируем $c$, у нас $c$ отображается в функцию $f_c$, которая в свою очередь есть функция $B \to A$, что и написано слева, ура.

Зачем же нам всё это? Ну допустим хотим узнать, чем разно $\omega^c$ ($\omega $ --- мощность счётного множества, $c$ --- континуального). Т.е. это есть $f : \mathbb{R} \to \mathbb{N}$. Знаем, что $\omega^c \le c^c = (2^{\omega})^c = 2^{\omega \times c} \le 2^{c \times c} = 2^{c}$, но, в свою очередь $\omega ^ c \ge 2^c$, т.е. искомое множество зажато между $2^c$ и $2^c$

Ну или ещё вариант --- $c^{\omega} = (2^{\omega})^{\omega} = 2^{\omega \times \omega} = 2^{\omega} = c$, но при этом $c^{\omega} \ge 2^{\omega}$, снова зажали.

