\Subsection{Равномощные множества и Счётные множества}
...тут я ещё не начинал записывать
\begin{definition}
    Множества $A$ и $B$ равномощны, если $\exists$ биекция между $A$ и $B$ 
\end{definition}
\begin{remark}
	Равномощность является отношением эквивалентности. Очев.
\end{remark}
Возможная, но не совсем корректная трактовка --- равномощные множества --- содержащие равное количество элементов. Для конечных это действительно верно, а с бесконечными уже не совсем понятно, сколько в них элементов. <<Бесконечность>>?

\begin{definition}
	Множество называется счётным, если оно равномощно множеству натуральных чисел.
\end{definition}

Пояснение: мы считаем натуральными числами множество $\{1, 2, 3, \cdots\}$, но никакой разницы с $\{0, 1, 2, 3, \cdots\}$ нет, т.к. существует биекция из одного в другое --- $i \to i - 1$.

\begin{example}
	Чётные числа --- счётное множество. Биекция --- $i \to 2 \times i$
\end{example}

\begin{lemma}
$A, B$ --- счётны $\Rightarrow$ $A \cup B$ --- счётно при $A \cap B = \emptyset$.
\end{lemma}
\begin{proof}
	$\exists f: N \to A \Rightarrow$ $A : \{a_1, a_2, a_3, \cdots\}$

	Аналогично $B: \{b_1, b_2, b_3, \cdots\}$

	Тогда запишем:

	$C: \{a_1, b_1, a_2, b_2, \cdots\}$, $c_{2i - 1} = a_i, c_{2i} = b_i$
\end{proof}
\begin{consequence}
$\mathbb{Z}$ --- счётно.
\end{consequence}
\begin{proof}
	$\mathbb{Z} = \{0, 1, 2, \cdots\} \cup \{-1, -2, -3, \cdots\}$, первое равномощно $\mathbb{N}$, как и второе, а значит $\mathbb{Z}$ --- счётно по предыдущей лемме.
\end{proof}
\begin{lemma}
	$B$ --- счётное, $A \subset B$, тогда $A$ --- конечное или счётное.
\end{lemma}
\begin{proof}
	$B$ ---  счётно, тогда можно записать $B: \{b_1, b_2, \cdots\}$

	Раз $A$ подмножество, то просто часть элементов отсутствует. Тогда мы пойдём сопоставлять числа оставшимся элементам. Первое оставшееся --- 1, второе --- 2, и т.д. Тогда или в какой-то момент у нас закончатся оставшиеся числа, т.е. найдётся то, после которого нет оставшихся, и тогда $A$ --- конечно, или мы построим биекцию между $A$ и натуральными числами. Это биекция, поскольку это инъекция и сюръекция (мы каждому натуральному поставили число, и всем элементам $A$ что-то одно сопоставили)
\end{proof}

\begin{theorem}[Лемма]
Любое бесконечное множество содержит счётное подмножество.
\end{theorem}
\begin{definition}
	Множество $X$ бесконечно, если $\forall i \in \mathbb{N}$ можно найти $i$ различных элементов из $X$.
\end{definition}
\begin{proof}
	Возьмём элемент из $X$. Назовём его $a_1$. Если в $X$ не осталось элементов, значит в нём был всего один элемент. Иначе возьмём из $X$ какой-то другой элемент, назовём его $a_2$. Если снова не осталось, то было всего два элемента. И так далее, построили $Y = \{a_1, a_2, a_3, \cdots\}$, и $Y \subset X, f: \mathbb{N} \Leftrightarrow Y$.
\end{proof}
\begin{example}
	То, что мы нашли счётное подмножество, никак не доказывает счётность исходного множества. Например, мы могли в своём процессе вытащить далеко не все элементы $X$:
	Счётное множество, для которого мы таким процессом не докажем счётность: $X = \{1, 2, 3, \cdots\}$, а выбрали из него --- $Y = \{2, 4, 6, \cdots\}$. 
\end{example}
\begin{example}
	Множество $(0, 1)$ не является счётным. Пока что, видимо, без доказательства.
\end{example}
\begin{consequence}
Если $A_1, A_2, A_3, \cdots, A_k$ --- счётны, то $A_1 \cup A_2 \cup \cdots A_k$ --- счётно
\end{consequence}
\begin{proof}
    Для дизъюнкных уже понимаем: последовательно объединим первое со вторым, результат с третьим, и так далее до $k$-го. А для недизъюнктных:

    Посмотрим на $A_1$ и $A_2 \setminus A_1$. Оба счётны, а тогда $A_1 \cup A_2 = A_1 \sqcup (A_2 \setminus A_1)$, где оба счётные.

    Теперь воспользуемся индукцией по $k$:
    База: $A_1, A_2$ счётны по условию, тогда $A_1 \cup A_2$ тоже счётно.
    Тогда $(A_1 \cup A_2 \cup A_3 \cdots A_k) = ((\cdots((A_1 \cup A_2) \cup A_3) \cup \cdots) \cup A_k)$, т.е. по сути то же, что и для дизъюнктных. Мы просто свели пересекающиеся к непересекающимся, на самом деле.
\end{proof}
\begin{lemma}
	$A_1, A_2, \cdots$ --- счётное число счётных множеств, т.е.  $\forall i \in \N \exists A_i  \text{--- счётное множество}$.

Тогда $A_1 \cup A_2 \cup \cdots$ тоже счётно.
\end{lemma}
\begin{proof}
	$A_1$ счётно, тогда $A_1: \{a_{11}, a_{12}, a_{13}, \cdots\}$. 
	Аналогично $A_2: \{a_{21}, a_{22}, a_{23}, \cdots\}$
	И так далее ещё счётное число строк.

	Теперь нам нужно эту таблицу представить в виде последовательности.
	Будем ходить по диагоналям: $a_{11}, a_{12}, a_{21}, a_{13}, a_{22}, a_{31}, \cdots$

	Утверждение --- любой элемент будет выписан. Рассмотрим элемент множества $i$ номер $j$, тогда оно будет на $i+j$-ой диагонали, а значит его номер точно не будет превышать $(i+j)^2$. Тогда получаем, что любой элемент будет выписан.

	Это всё для непересекающихся множеств, а для пересекающихся --- давайте просто не выписывать элементы, которые уже выписали. 
\end{proof}
\begin{exerc}
	$\mathbb{N}, \mathbb{Z}$ --- счётны. $[0, 1)$ --- несчётно (просто знаем), знаем $\mathbb{R}$ --- несчётно. $\mathbb{Q}$ --- счётно или нет?
\end{exerc}

\begin{proof}
	$\mathbb{Q}_+$ счётно. Давайте представим его в виде $A_1 \cup A_2 \cup \cdots$, где $A_i = \{\frac{m}i | m \in \mathbb{N}\}$. Так получили, что $\mathbb{Q}_+$ это ровно счётное число счётных множеств. А тогда мы получаем (по предыдущей теореме), что $\mathbb{Q}_+$ счётно. Аналогично получим, что $\mathbb{Q}_-$ тоже счётно, добавим ноль, объединим полученное (два счётных множества и ноль) и получим, что $\Q$ --- счётно, ура.
\end{proof}
\begin{lemma}
	$A, B$ --- счётны, тогда $A \times B$ --- счётно.
\end{lemma}
\begin{proof}
	$A, B$ --- счётны, тогда $A: \{a_1, a_2, \cdots\}$, $B: \{b_1, b_2, \cdots\}$
	Элементы из $A \times B$ выглядят так: $(a_i, b_j)$, тогда давайте запишем следующее:

	$A_1 = a_1 \times B = \{(a_1, b_1), (a_1, b_2), (a_1, b_3), \cdots\}$, $A_2$ аналогично, и так далее.
	Тогда каждое $A_i$ --- счётно, и их счётное число, значит их объединение, которое и есть $A \times B$ счётно, по доказанной лемме.
\end{proof}

\Subsection{Несчётные множества}

\begin{lemma}
Пусть $A$ бесконечно, а $B$ --- конечно. Тогда $A \cup B$ равномощно $A$.
\end{lemma}
\begin{proof}
	$B$ заменим на $B' = B \setminus A$. $B'$ или станет пустым, или останется конечным.

	Очевидно, что $A \cup B = A \sqcup B'$

	У $A$ есть счётное подмножество $A' = \{a_1, a_2, \cdots\}$, тогда $A = (A\setminus A') \cup A'$.

	Хотим построить биекцию между $A = A' \cup (A\setminus A')$ и $A' \cup B' \cup (A\setminus A')$

	Часть <<$(A \setminus A')$>> есть в обеих половинах, а значит между $ (A \setminus A') $ слева и $ ( A \setminus A')$ справа построим тождественную биекцию. Из оставшегося $a_i \in A'$ будем отображать в $b_i \in B'$, если $i \le k$, или в $a_{i-k}$ при $i > k$. Понятно, что это биекция. Таким образом мы возьмём все элементы как из $B$, так и из $A'$.
\end{proof}

\begin{remark}
	Доказательство можно модифицировать для случая, когда $B$ счётно. Тогда давайте на последнем шаге чётные индексы отображать в $a_{\frac{i}2}$, а нечётные --- в $b_{\frac{i-1}2}$. 
\end{remark}
\begin{theorem}
	Множество $X$ последовательностей (бесконечных) из нулей и единиц не счётно. (Бинарные строки бесконечной длины)
\end{theorem}
\begin{proof}
	От противного: пусть счётен, тогда есть биекция $f \colon \mathbb{N} \to X$. Тогда выпишем последовательности $f(1), f(2), f(3), \cdots$. 
	А теперь воспользуемся диагональным (методом Кантора): Посмотрим на элемент $a_{11}$, возьмём элемент $1 - a_{11}$. Затем на элемент $a_{22}$, возьмём $1 - a_{22}$. И так далее, строим последовательность $1 - a_{ii}$. Получили бесконечную последовательность нулей и единиц, значит она элемент $X$. Но при этом она не может быть любой $i$-ой последовательностью, поскольку её $i$-ый элемент не совпадает с $i$-ым элементом строки $i$ по тому, как мы строили нашу последовательность.
	Противоречие. Значит мы не можем вот так вот выписать наши элементы $X$, значит биекции $f$ не существует. 
\end{proof}
\begin{consequence}
	Множество чисел из отрезка $[0, 1]$ несчётно.
\end{consequence}
\begin{proof}
	Покажем равномощность множеству $X$ из теоремы. Из бесконечной последовательности число получить легко --- припишем слева <<$0,$>>, а все элементы последовательности запишем слитно. Могло показаться, что получили биекцию, но нет. У нас разные последовательности могут соответствовать одному числу --- $0,100000000\cdots$ и $0,011111111\cdots$  --- разные последовательности, но являются одним числом. 
	Возьмём две последовательности --- $0,a_{11}a_{12}a_{13}\cdots$, $0,a_{21}a_{22}a_{23}\cdots$. Утверждение --- они задают одно число тогда и только тогда, когда они имеют совпадающий префикс, а затем у одного числа идёт единица и после только нули, а у второго ноль и затем только единицы.
	Докажем от противного --- пойдём по таким двум числам слева направо и найдём первый момент, когда они отличаются. В одном ноль, во втором единица. Далее всё идёт сколько-то, как мы предсказали (там где ноль --- единицы, где единица --- нули), затем, раз от противного, что-то пойдёт не так. Тогда в этот момент расхождения с предсказанием можно посмотреть на текущие величины чисел, образованные пройденными префиксами (до момента расхождения с предсказанием включительно) и можно заметить, что на этот момент числа отричаются настолько, что дальше мы даже бесконечностью цифр не сможем сделать расстояние между двумя числами сделать сколь угодно близким к нулю. 
	А числа такого вида это просто суммы некоторых конечных количеств отрицательных степеней двойки, что есть подмножество $\Q$, а значит их счётное число. Можно записать, словно это $\Q \cap [0, 1]$ и ничего не потерять
	Тогда $X \sim [0, 1] \sqcup (\mathbb{Q} \cap [0, 1])$. Результат пересечения счётен, а значит объединение равномощно бесконечной левой части, т.е. $X \sim [0, 1]$
\end{proof}
Теперь знаем, что натуральные счётны, чётные счётны, целые счётны, рациональные положительные счётны, просто рациональные счётны. А вот вещественные ($\R$) уже несчётны, т.к. содержат $[0, 1]$.

\begin{example}
	Множество точек границ треугольника и вписанного круга равномощны, т.к. можно построить биекцию из центра. 
\end{example}

\begin{theorem}
	$X = [0, 1]$ и $Y = [0, 1] \times [0, 1]$ равномощны (т.е. отрезок равномощен квадрату).
\end{theorem}
\begin{proof}
	Рассмотрим произвольный элемент из $Y$. Он представляет из себя пару чисел из $[0, 1]$, т.е. две бесконечные битовые строки --- $x_i$ и $y_1$. Тогда давайте построим из этих двух битовых строк новую, забирая элементы из исходных поочерёдно --- $x_1 y_1 x_2 y_2 \cdots$. Это будет новая битовая строка, так что она лежит в $[0, 1]$, но при этом она однозначно соответствует исходной паре битовых строк $x_i$ и $y_i$. Таким образом мы получили биекцию между множеством пар битовых строк и просто множеством битовых строк, а значит искомые множества равномощны, что и требовалось доказать.
\end{proof}

Теперь давайте вернёмся к теореме о несчётности отрезка $[0, 1]$ и докажем это снова, но уже другим способом.

\begin{proof}
	Возьмём множество вложенных отрезков $I_1 \supset I_2 \supset I_3 \supset \cdots, \forall i \in \N I_i \subset [0, 1]$. Из матанализа знаем, что пересечение всех этих отрезков непусто. 

	Теперь предположим, что $[0, 1]$ --- счётно. Тогда занумеруем все его элементы --- $a_1, a_2, \cdots$. 

	Теперь будем строить систему отрезков. Разобьём $[0, 1]$ на три части. Заметим, что $a_1$ может лежать не больше чем в двух из них, а значит найдётся часть, в которой нет точки $a_1$. Положим эту часть за $I_1$. $I_1$ в свою очередь тоже разобьём на три части, и положим за $I_2$ ту из них, в которой не лежит точка $a_2$. И так далее, получим систему вложенных отрезков, причём $a_j \not \in I_j$.

	Мы знаем, что для множества вложенных отрезков (которые являются подотрезками $[0, 1]$) существует точка, лежащая во всех них. Назовём её $x$. Но тогда, раз мы занумеровали все точки $[0, 1]$, то $x = a_j$, но $a_j \not \in I_j$, а значит и не лежит в пересечении, противоречие.
\end{proof}

Теперь немного отойдём, видимо, в сторону:

\begin{theorem}[Кантора-Бернштейна]
	Пусть $A, B$ --- множества, и существуют $f : A \to B$ --- инъекция и $g : B \to A$ --- тоже инъекция. Тогда существует биекция $h$ между $A$ и $B$ (и, как следствие, множества равномощны)
\end{theorem}
\begin{proof}
	Давайте построим ориентированный двудольный граф. Вершины левой доли --- элементы множества $A$, левой --- множества $B$. Проведём рёбра соответствующие отображениям ($f(a_1) = b_1$ --- проводим ребро из $a_1$ в $b_1$, $g(b_2) = a_2$ --- проводим ребро из $b_2$ в $a_2$). Тогда у нас из каждой вершины точно выходит ребро, и в каждую вершину входит не больше одного ребра (т.к. инъекции).

	Что может быть в полученном графе?
	\begin{enumerate}
		\item{Циклы}
		\item{Цепочки с началом, но без конца}
		\item{Цепочки без начала и без конца}
	\end{enumerate}
	Цепочек с концом быть не может, т.к. из каждой вершины всё же есть ребро, значит всегда есть, куда идти (ну или замыкаться).

	Теперь построим биекцию. Все циклы у нас бывают только чётной длины, т.к. граф двудольный, а значит их можно разбить на пары, которые и положим в биекцию. Для второго случая просто сопоставим первый элемент со вторым, третий с четвёртым и т.д. А для третьего --- ткнём в случайное место, и пойдём разбивать на пары, но уже в обе стороны одновременно. 

	Каждый элемент попадает ровно в один случай из трёх, а значит попадёт в какую-то пару. Вот и биекция. 
\end{proof}

Вам доказательство показалось каким-то странным? Хорошо, держите ещё одно:

\begin{proof}
	Положим $A_0 = A, A_1 = g(B) \subset A_0$. Так мы получаем, что $g$ есть биекция между $A_1$ и $B$. Положим $A_2 = g(f(A)) \subset g(B) = A_1$. Так получили $A_2 \subset A_1 \subset A_0$, а ещё $g(f(A))$ --- биекция между $A_0$ и $A_2$. Да и вообще положим $h = g \circ f$. Это будет биекция между $A_i$ и $A_{i + 2}$. Так то нам нужно доказать, что $A_0$ равномощно $A_1$, и тогда мы победим, поскольку $A_0 = A$, а $A_1$ равномощно $B$. 

	Введём $C_i = A_i \setminus A_{i + 1}$. Заметим, что $C_i$ равномощно $C_{i+2}$, поскольку $C_{i+2} = A_{i + 2} \setminus A_{i + 3}, C_i = A_i \setminus A_{i+1}$, а $A_i$ и $A_{i+2}$ равномощны (т.к. есть биекция --- $h$), ровно как равномощны и $A_{i+1}$ с $A_{i + 3}$ (по той же причине). 

	Положим $C = \bigcap A_i$. Тогда:

	{
	$$
	A_0 = \overbrace{(A_0 \setminus A_1)}^{=C_0} \cup \overbrace{(A_1 \setminus A_2)}^{=C_1} \cup \overbrace{(A_2 \setminus A_3)}^{=C_2} \cup \cdots \cup C
	$$
	$$
	A_1 = \underbrace{(A_1 \setminus A_2)}_{=C_1} \cup \underbrace{(A_2 \setminus A_3)}_{=C_2} \cup \cdots \cup C
	$$
	}

	Теперь отобразим частями: $C$ в $C$, $C_0$ в $C_2$, $C_1$ в $C_3$ и так далее. Мы знаем, что они в парах равномощны, и так мы в итоге построили биекцию, т.к. каждый объект попал в какую-нибудь часть. Всё, ура, $A_0$ и $A_1$ равномощны, а значит равномощны и $A$ с $B$, т.е. есть биекция, ура!
\end{proof}

\begin{theorem}[Теорема Кантора]
	Пусть $X$ --- множество, а $2^X$ --- множество всех подмножеств $X$. Тогда $X$ и $2^X$ не равномощны.
\end{theorem}
\begin{proof}
	Для конечных это верно, поскольку $\forall k \in \N \cup \{0\}  2^k > k$. 

	Теперь от противного --- пусть $\exists \varphi : X \to 2^X$ и $\varphi$ --- биекция. Возьмём $Z = \{ x \in X \colon x \not \in \varphi(x) \}$. $Z$ --- какое-то подмножество $X$, а значит, т.к. $\varphi$ --- биекция, то $\exists z \in X : \varphi(z) = Z$. Посмотрим как расположено $z$ относительно $Z$: если $z \in Z$, то $z \not \in \varphi(z)$, т.е. $z \not \in Z$, противоречие (т.к. все переходы двусторонние).
\end{proof}

\begin{definition}
	Континуальное множество --- множество, равномощное $[0, 1]$. (Ещё возможный вариант: континум --- отрезок $[0, 1]$. Или $\R$. Какая разница, они всё равно равномощны).
\end{definition}

Заметим, что мы всегда можем <<увеличивать>> множество: $\N \to 2^{\N} \to 2^{(2^{\N})} \to \cdots$ . Как итог --- нет <<наибольшего>> множества.

\begin{theorem}
	Не существует множества всех множеств
\end{theorem}
\begin{proof}
	От противного --- пусть существует, обзовём его за $X$. Тогда рассмотрим $2^X$. Тогда $2^X \in X$ (т.к. $X$ это уже множество всех множеств), но $|X| < |2^X|$, противоречие.
\end{proof}

\begin{theorem}
	Дизъюнктное объединение двух континуальных множеств континуально
\end{theorem}
\begin{proof}
	$A_1, A_2$ изоморфны множеству бесконечных битовых строк. Тогда их объединение будет тоже изоморфно множеству бесконечных битовых строк, поскольку можно просто для любого элемента, если он из $A_1$ взять его битовую запись и приписать в начало ноль, а если он из $A_2$, то приписать в начало единицу. Всё, получили все битовые строки и записи для всех элементов, значит объединение континуально.
\end{proof}

\begin{theorem}
	Объединение счётного числа континуальных множеств континуально.
\end{theorem}
\begin{proof}
	Пусть у нас есть набор множеств $A_i$. Знаем, что $A_i$ изоморфен отрезку $[0, 1]$. Тогда давайте просто расположим эти отрезки параллельно друг другу внутри квадрата $[0, 1] \times [0, 1]$. Мы так можем сделать поскольку можем взять какой-нибудь столбец для $A_1$, $A_2$ и т.д., иначе говоря, выбрать счётное подмножество из отрезка $[0, 1]$. Всё, тогда имеем $[0, 1] ~ A_0$, и $A_0 \subset A_0 \cup A_1 \cup A_2 \cup \cdots \subset [0, 1] \times [0, 1] ~ [0, 1]$. Тогда мощность объединения находится между мощностями $[0, 1]$ и $[0, 1] \times [0, 1] \times [0, 1]$. Но мощность правого равна мощности левого, а значит искомое объединение --- континуально.
\end{proof}

\begin{remark}
	Мы не знаем, есть ли что-то больше счётного, но меньше континуального. (Если быть точным, то знаем, что в $ZF(C)$ нельзя доказать ни отсутствие там чего-либо, так и присутствие, но забейте).
\end{remark}

\Subsection{Об операциях над мощностями}

Если хотим сложить множества (мощности), то нам нужна мощность следующего множества

{
\large
$$
A \times \{0\} \cup B \times \{1\}
$$
}

О корректности --- если выбирать разные множества одной мощности, то можно построить биекцию и не париться.

Коммутативность очевидна, т.е. мощность суммы не изменится от перестановки $A$ и $B$.

Произведение мощностей, ожидаемо, есть мощность множества $|A \times B|$

С возведением в степень чуть сложнее: пусть $|A| = a, |B| = b$, то $a^b = |A^B|$, где последнее --- множество всех функций, действующих из $B$ в $A$.

Хотим проверить, что $A^{B \sqcup C} = A^B \times A^C$. Имеем $g : B \to A, h : C \to A$, и функция $f : B \cup C \to A$ взаимооднозначно определяет $g $ и $h$.

Теперь хотим проверить, что $(ab)^c = a^c \times b^c$. Слева имеем $\{f : C \to A \times B\}$, а справа $\{ f : C \to A  \} \times \{g : C \to B\}$. Но тогда заметим, что функций из первого множества есть две координаты в образе и можно рассмотреть проекции образов  на $A$ и на $B$ как две функции, и всё хорошо.

Остаётся $(a ^ b)^c = a^{b \times c}$. По сути $a^{b \times c}$ это $\{f | f : B \times C \to A\}$. Что плюс-минус есть $f_c(x) = f(x, c)$ --- как только мы фиксируем $c$, у нас $c$ отображается в функцию $f_c$, которая в свою очередь есть функция $B \to A$, что и написано слева, ура.

Зачем же нам всё это? Ну допустим хотим узнать, чем разно $\omega^c$ ($\omega $ --- мощность счётного множества, $c$ --- континуального). Т.е. это есть $f : \mathbb{R} \to \mathbb{N}$. Знаем, что $\omega^c \le c^c = (2^{\omega})^c = 2^{\omega \times c} \le 2^{c \times c} = 2^{c}$, но, в свою очередь $\omega ^ c \ge 2^c$, т.е. искомое множество зажато между $2^c$ и $2^c$

Ну или ещё пример --- $c^{\omega} = (2^{\omega})^{\omega} = 2^{\omega \times \omega} = 2^{\omega} = c$, но при этом $c^{\omega} \ge 2^{\omega}$, снова зажали.

