\Subsection{Про отношения}

\begin{definition}
Пусть есть множество $A$, и на нём  ввели отношение $R \subset A \times A$ (Напоминание --- отношение есть подмножество $A \times A$).

$R$ --- отношение эквивалентности, если оно удовлетворяет следующим аксиомам:

\begin{enumerate}
\item{$a \in A (a, a) \in R$} --- рефлексивность
\item{$a, b \in A, (a, b) \in R \to (b, a) \in R$} --- симметричность
\item{$a, b, c \in A, (a, b), (b, c) \in R \to (a, c) \in R$ --- транзитивность}
\end{enumerate}
\end{definition}

Это хорошо, но нам оно не нужно, какая досада.

\begin{definition}

	Отношение $R'$ называется отношением порядка, если оно удовлетворяет следующим аксиомам:

\begin{enumerate}
\item{$a \in A (a, a) \in R$} --- рефлексивность
\item{$a, b \in A, (a, b) \in R, (b, a) \in R \to a = b$} --- антисимметричность
\item{$a, b, c \in A, (a, b), (b, c) \in R \to (a, c) \in R$ --- транзитивность}
\end{enumerate}
\end{definition}

Обычно для него используют значок, например $\le$. Можно рисовать более закорючно, но мне влом.

\begin{definition}
	Частично Упорядоченное Множество --- ЧУМ --- пара $(X, \le)$ --- множество, и отношение порядка на нём.
\end{definition}

Теперь про частично упорядоченное множество. Возьмём  $(X, \le)$, всё, получили ЧУМ. Простейшие примеры --- натуральные, рациональные, действительные числа и операция меньше-или-равно. Можно взять тривиальное отношение --- в нём находятся только пары вида $(a, a), a \in X, $ т.е. $a \le'' b : a = b$. Ещё отношение порядка --- рассмотрим $2^X$, с отношением <<являться подмножеством>>. Аксиомы, очевидно, выполняются. 

Рассмотрим функции $f : \mathbb{R} \to \mathbb{R}$ и введём отношение порядка $f \le g \Leftrightarrow \forall x \in \mathbb{R} f(x) \le g(x)$. Тут уже явно видно, что бывают несравнимые элементы. Тут, например, это пары функций, у которых на первой половине первая больше второй, а на второй половине --- вторая больше первой.

Можно строить по $\le$ и отношение строгого порядка $<$ --- $x < y \Leftrightarrow x \le y, x \neq y$. Его аксиомы:


\begin{enumerate}
\item{$a \in A (a, a) \not\in R$} --- антирефлексивность
\item{$a, b, c \in A, (a, b), (b, c) \in R \to (a, c) \in R$ --- транзитивность}
\end{enumerate}

А теперь давайте размножать ЧУМы. 

Пусть $X, Y$ --- ЧУМы. Тогда можно строить:

\begin{enumerate}
	\item{$X \sqcup Y$ --- внутри одной доли используется старое отношение, а элементы из разных долей просто несравнимы}
	\item{$X + Y$ --- считаем, что любой элемент $X$ $\le$ любого элемента $Y$. Пример --- возьмём натуральные числа и ещё раз натуральные числа. Тогда $5 < 6 < \cdots < 1' < 2'$, например.}
	\item{$X \times Y$. Есть два варианта --- покоординатно --- $(x_1, y_1) \le (x_2, y_2) \Leftrightarrow x_1 \le x_2 $ И $y_1 \le y_2$. Второй вариант --- лексикографически --- $(x_1, y_1) \le (x_2, y_2) \Leftrightarrow (x_1 < x_2)$ ИЛИ $((x_1 = x_2)$ И $ (y_1 \le y_2))$ }
\end{enumerate}

\begin{definition}
	ЧУМ --- линейный, если $\forall$ два элемента сравнимы. 
\end{definition}

\begin{definition}
Максимальный элемент --- тот, больше которого нет. Наибольший элемент --- который больше либо равен всех остальных. 
\end{definition}

Рассмотрим все подмножества трёхэлементного множества $\{a, b, c\}$. Можно нарисовать картинку, как они расположены, но я пока не гений картинок. В общем, если $x$ наибольший, то он максимальный. В обратную сторону далеко не всегда верно, т.к. иногда можно быть несравнимым с кем-нибудь. 

\begin{definition}
	$X, Y$ --- ЧУМы, $\varphi: X \to Y$ --- изоморфизм ЧУМов, тогда и только тогда, когда $\varphi$ --- биекция, и $a \le b \Leftrightarrow \varphi(a) \le \varphi(b)$
\end{definition}

Пусть ещё пример --- $k \le n$ --- $k$ делителеь $n$. И тут мы вспомним, что можно сужать порядок --- резать его множество. Давайте сузим наш порядок на множество $\{1, 2, 3, 6\}$ --- порядок делимости.

А ещё давайте рассмотрим $2^{<a, b>}$ --- $\{\varnothing, \{a\}, \{b\}, \{a, b\}\}$. Можно показать, что оно изоморфно предыдущему множеству. Пример изоморфных есть, какие примеры не изоморфны?

Изоморфно ли $\mathbb{N}, \le$ и $\mathbb{R}, \le$? Нет, т.к. биекции точно нет, т.к. по мощностям не сходятся множества.
Изоморфно ли $\mathbb{N}, \le$ и $\mathbb{Z}, \le$? Нет, т.к. во втором множестве нет наименьшего элемента, а в первом --- есть, а наименьший должен переходить в наименьший (выводится довольно понятно, как) 

Изоморфно ли $\mathbb{Z}, \le$ и $\mathbb{Q}, \le$? Нет, т.к. давайте возьмём 0 и 1 из $\mathbb{Z}$ и отобразим их куда-либо, получили $\varphi(0), \varphi(1)$. Между ними где-то есть $\frac{\varphi(0) + \varphi(1)}{2}$. Подействуем на него $\varphi^{-1}$, получим, что прообраз должен жить в $\mathbb{Z}$ между 0 и 1. Но там никого нет! Значит наше предположение о существовании биекции неверно.

Рассмотрим $(\mathbb{Z}, \le)$ и $(\mathbb{Z} + \mathbb{Z}, \le)$ Рассмотрим 0 и $0'$ во втором, между ними находится бесконечное число элементов. Но рассмотрев прообразы, это какие-то два целых, между ними вся бесконечность должна будет поместиться, но нет, т.к. между двумя целыми конечное число элементов.

\begin{definition}
$x, y$ --- соседние, если $x \le y$ и между $x$ и $y$ нет элементов и порядок линейный.
\end{definition}
\begin{definition}
Линейный порядок называется плотным, если $\forall x, y, x < y \to \exists z : x < z < y$
\end{definition}

\begin{theorem}
Пусть $X, Y$ --- ЧУМы. Тогда если они конечные, с линейными порядками, то они изоморфны тогда и только тогда, когда $|X| = |Y|$. 
\end{theorem}
\begin{proof}
Будем брать наименьшие элементы и попарно отображать их друг в друга и удалять. Отношение относительно минимальных и прочих сохраняется, а на меньшем можем построить дальше по индукции. А если размеры не равны, то мы умерли ещё на этапе биекции.
\end{proof}
\begin{remark}
	Важна конечность! Для просто равномощных бесконечных мы уже видели контрпримеры. 
\end{remark}

\begin{theorem}
	Пусть $X, Y$ --- счётные ЧУМы, имеют плотный и линейный  порядок, и в $X$ и $Y$ нет наибольшего и наименьшего элементов, тогда $X$ изоморфно $Y$. (такие ЧУМы существуют, например $(\mathbb{Q}, \le)$)
\end{theorem}
\begin{proof}
	Выпишем подряд $X --- \{x_i\}$, $Y --- \{y_i\}$. Отобразим $x_1$ в $y_1$. А дальше есть $x_2$, хотим отобразить куда-то. Отобразим $x_2$ в такой элемент, который относительно $y_1$ расположен так же, как $x_2$ расположен относительно $y_1$.
	Такой найдётся, т.к. нет минимума и максимума. После этого аналогично поступим с $y_2$ (второй в выписанном списке игреков, если он ещё не взят). Затем так же поступим с $x_3$, но теперь уже смотрим на отношения $x_3$ с $x_1, x_2$. Это получится, т.к. нет минимума, максимума и ещё мы плотны. Затем на $y_3$ и т.д.

	Так мы задействуем все элементы из $X$ и $Y$, и при этом ничего не сломаем из ограничений на отношение порядка, а значит получим изоморфизм ЧУМов, ура. 
\end{proof}
