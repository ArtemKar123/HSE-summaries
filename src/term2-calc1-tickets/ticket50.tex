%BEGIN TICKET 50
\begin{definition}
    $X, Y$ --- векторные пространства,  

    $A\!: X \to Y$ --- линейный оператор, если $\forall x, y \in X, \forall \alpha, \beta \in \R A(\alpha x + \beta y) = \alpha A(x) + \beta A(y)$.
\end{definition}
\begin{properties}
\begin{enumerate}
    \item $A 0_X = 0_Y$. Доказательство: $\alpha = 0, \beta = 0$.
    \item $A (\sum\limits_{k=1}^n \lambda_k x_k) = \sum\limits_{k=1}^n \lambda_k A(x_k)$. Доказательство: индукция.
\end{enumerate}
\end{properties}
\begin{definition}
    $A, B$ --- линейный оператор:  $X \to Y$.

     $(A+B)(x) \coloneqq A(x) + B(x)$.

      $(\lambda A)(x) = \lambda A(x)$. 

      То есть получили векторное пространство линейных операторов.
\end{definition}
\begin{definition}
    $A\!: X \to Y, B\!: Y \to Z$ --- линейные операторы  $B \circ A\!: X \to Z$.  $(B \circ A)(x) \coloneqq B(A(x))$.
\end{definition}
\begin{remark}
    Это линейный оператор.
\end{remark}
\begin{definition}
    Обратный оператор: $A\!: X \to Y$,  $B\!: Y \to X$ обратный к  $A$, если  $A \circ B = Id_Y$ и  $B \circ A = Id_x$. Обозначается $A^{-1}$.
\end{definition}
\begin{properties}
    \begin{enumerate}
        \item Если обратный оператор $\exists$, то он единственный.
        \item  $(\lambda A)^{-1} = \frac{1}{\lambda} A^{-1}$.
        \item $A\!: X \to X$ --- обратимые операторы образуют группу по операции композиции.
    \end{enumerate}
\end{properties}
\begin{proof}
    \begin{enumerate}
        \item $B \circ A = Id_X \implies$ A --- инъекция. Если  $A(x) = A(y) \implies x = B(A(x)) = B(A(y)) = y$.

             $A \circ B = Id_Y \implies$  $A$ --- сюръекция.  $A(B(y)) = y \implies$ просто биекция. 

             Пусть  $B, C$ --- обратные к  $A$.  $B(A(x)) = B \circ A(x) = x = C \circ A(x) = C(A(x)) \implies B = C$ на множестве значений $A$, но $A$~---~сюръекция.
         \item $((\frac{1}{\lambda}A^{-1}) \circ (\lambda A))(x) = \frac{1}{\lambda}A^{-1}(\lambda A(x)) = x$.
    \end{enumerate}
\end{proof}
\begin{example}[для 3 свойства]
    $X = \R^n, Y = R^m$. Можно рассматривать линейные операторы как матрицы $\Rightarrow Ax = y$ ($m$ на $n$ матрица).
\end{example}

\begin{definition}[Матричная запись]
    $A\!: R^n \to R^m$. Возьмем базисный вектор  $e_k$ --- везде 0, кроме  $k$-ой позиции~---там 1.

    Пусть  $x = \sum\limits_{i=1}^n x_ie_i$. Тогда  $Ax = A(\sum\limits_{k=1}^n x_k e_k) = \sum\limits_{k=1}^n x_k \underbrace{A_{e_k}}_{\coloneqq A_k \in \R^m}$. 

    То есть получили набор столбцов. Из которого можно получить матрицу.
%END TICKET 50