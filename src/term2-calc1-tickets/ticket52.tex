%BEGIN TICKET 52
\begin{theorem}
    $A\!: X \to Y$ --- линейный оператор. Тогда

     $\|A\| = \sup\limits_{\| x\|_x < 1} \|Ax\|_Y = \sup\limits_{\| x\|_x = 1} \|Ax\|_Y = \sup\limits_{x \neq 0} \frac{\|Ax\|_Y}{\|x\|_x} = \inf\{ c > 0 \mid \|A_x\|_Y \le C \|x\|_X\}$.
\end{theorem}
\begin{proof}
    Обозначим за $N_i$  $i$-ый элемент этой цепочки. 

     $N_1 \ge N_2$ и $N_1 \ge N_3$,  так как $N_2, N_3 \subset N_1$. 

     $N_3 \ge N_4$. $\frac{\|Ax\|_Y}{\|x\|_X} = \frac{1}{\|x\|}\|Ax\|_Y = \|A \frac{x}{\|x\|}\|_X \le N_3$.

     $N_4 = N_5$. $N_5 = \inf \{ c>0 \mid \frac{\|Ax\|_Y}{\|x\|_X} < c\}$

     Теперь докажем, что $N_1 \le N_2$. Пусть $\| x \| \le 1 \implies \| (1-\eps) x \| < 1 \implies \| A((1-\eps)x) \| \le N_2$. Воспользуемся линейностью $A$: вытащим  $(1-\eps)$ за скобку. После этого устремим  $\eps$ к 0. Тогда  $\| Ax \| \le N_2 \implies N_1 = \sup\limits_{\| x \| \le 1} \| A x\| \le N_2$.

     Теперь докажем, что $N_1 \le N_4$. $\| x \| \le 1$. Тогда $y \coloneqq \frac{x}{\| x \|}$, $\|y\| = 1 \implies \| A_y \| \le N_4 \implies \| Ax \| \le \frac{1}{\|x\|} \cdot \| Ax \| \le N_4 \implies \| A_x \| \le N_4 \implies N_1 = \sup\limits_{\|x \| \le 1} \|Ax\| \le N_4$.
\end{proof}
%END TICKET 52