%BEGIN TICKET 41
\begin{definition}
    $(X, \rho_X)$ и  $(Y, \rho_Y)$ --- метрические пространства, $E \subset X$.  $f\!: E \to Y$, \\$a$~---~предельная точка  $E$,  $b \in Y$.

     $b = \lim\limits_{x \to a} f(x)$ означает, что

     По Коши:  $\forall \eps > 0 \exists \delta > 0 \forall x\!: \rho_X(x,a) < \delta \land a \neq x \in E \implies \rho_Y(f(x), b) < \eps$.

     В терминах окрестностей: $\forall \eps > 0 \exists \delta > 0\!: f(\underbrace{\dot{B}_{\delta}(a)}_{\in X} \cap E) \subset \underbrace{B_\eps(b)}_{\in Y}$

     По Гейне: $\forall$ последовательности  $a \neq x_n \in E\!: \lim x_n = a \implies \lim f(x_n) = b$ единственный.
\end{definition}
\begin{theorem}
    Все определения равносильны. 
\end{theorem}
\begin{proof}
    Упражнение (смотри доказательство для функций).
\end{proof}
\begin{theorem}[Критерий Коши]
    $f\!:E \subset X \to Y$,  $Y$ --- полное,  $a$~---~предельная точка  $E$. Тогда

     $\exists \lim\limits_{x \to a} f(x) \iff \forall \eps > 0 \exists \delta > 0 \forall x, y \in \dot{B}_{\delta}(a) \cap E \implies \rho_Y(f(x), f(y)) < \eps$.
\end{theorem}
\begin{proof}
    $\Rightarrow$. Упражнение: взять доказательство и заменить модуль на  $\rho$.

     $\Leftarrow$. Проверим определение по Гейне. Надо доказать, что  $a \neq x_n \in E \land \lim x_n = a \implies \lim f(x_n)$ существует.

      $f(x_n)$ --- последовательность в  $Y$ --- полное. Поэтому достаточно проверить, что  $f(x_n)$~---~фундаментальная последовательность. Возьмем  $\eps > 0$, по нему  $\delta > 0$ из условия. По  $\delta > 0$ берем  $N$, такое что  $\forall n \ge N\!: \rho_X(x_n, a) < \delta \implies x_n \in \dot{B}_\delta(a)\cap E$ при $n \ge N \implies \forall m, n \ge N\!: \rho_Y(f(x_n), f(x_m)) < \eps \implies f(x_n)$ фундаментальная $\implies f(x_n)$ имеет предел.
\end{proof}

\begin{theorem}[об арифметических действиях с пределами]
    $f, g\!: E \subset X \to Y$,  $Y$ --- нормированное пространство,  $a$ --- предельная точка  $E$.\\
    Пусть  $\lim\limits_{x \to a} f(x) = b, \lim\limits_{x\to a} g(x) = c \land \alpha, \beta \in \R$. Тогда

     \begin{enumerate}
         \item $\lim\limits_{x \to a} \alpha f(x) + \beta g(x) = \alpha b + \beta c$.
         \item Если  $\lambda\!: E \to \R$, такое что  $\lim\limits_{x \to a} \lambda(x) = \mu \in \R$, то  $\lim\limits_{x \to a} \lambda(x) f(x) = \mu b$.
         \item  $\lim\limits_{x \to a} \lVert f(x) \rVert = \lVert b \rVert$
         \item Если  $Y$ --- пространство со скалярным произведением, то  $\lim\limits_{x \to a} \langle f(x), g(x) \rangle = \langle b, c \rangle$.
         \item Если $Y = \R$ и  $c \neq 0$, то  $\lim\limits_{x \to a} \frac{f(x)}{g(x)} = \frac{b}{c}$. 
    \end{enumerate}
\end{theorem}
\begin{proof}
    Проверка по Гейне. Берем $x_n \to a$,, тогда  $f(x_n) \to b, g(x_n) \to c$ и теорема про пределы последовательности.
\end{proof}
%END TICKET 41