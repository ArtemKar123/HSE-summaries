%BEGIN TICKET 43
\begin{theorem}[Непрерывный образ компакта --- компакт]
    $(X, \rho_X), (Y, \rho_Y)$ --- метрические пространства, $K \subset X$,  $K$ --- компакт.

    $f\!: K \to Y$ непрерывна во всех точках. Тогда  $f(K)$ --- компакт.
\end{theorem}
\begin{proof}
    Рассмотрим открытое покрытие $f(K) \subset \bigcup\limits_{\alpha \in I} U_\alpha$ --- открытые $\implies K \subset f^{-1}(\bigcup\limits_{\alpha \in I} U_\alpha) = \bigcup\limits_{\alpha \in I} f^{-1}(U_\alpha)$ по непрерывности $f$  $f^{-1}(U_\alpha)$ --- открыто  $\implies$ это открытое покрытие  $K$, но  $K$ --- компакт  $\implies$ выбираем конечное подпокрытие  $K \subset \bigcup\limits_{j=1}^n f^{-1}(U_{\alpha_j}) = f^{-1}(\bigcup\limits_{j=1}^n U_{\alpha_j}) \implies f(K) \subset \bigcup\limits_{j=1}^n U_{\alpha_j}$. Нашли конечное подпокрытие  $\implies f(K)$ --- компакт.
\end{proof}

\begin{definition}
    $f\!: E \subset X \to Y$ --- ограниченное отображение, если  $f(E)$ --- ограниченное множество.
\end{definition}
\begin{consequence}
    Непрерывный образ компакта замкнут и ограничен.
\end{consequence}
\begin{proof}
    Знаем, что непрерывный образ компакта --- компакт. А следовательно, образ замкнут и ограничен.
\end{proof}
\begin{consequence}
    Если $K$ --- компакт и  $f$ непрерывна на  $K$, то $f$ --- ограниченное отображение.
\end{consequence}
\begin{consequence}[Теорема Вейерштрасса]
    $f\!: K \to \R$,  $K$ --- компакт,  $f$ непрерывна на  $K$.

    Тогда  $\exists a,b \in K\!: f(a) \le f(x) \le f(b) \quad \forall x \in K$.
\end{consequence}
\begin{proof}
    $f(K)$ --- ограниченное множество в  $\R \implies B \coloneqq \sup\limits_{x \in K} f(x) \in \R \implies \exists x_n \in K\!: \lim f(x_n) = B$. При этом  $x_n \in K$ --- секвенциальный компакт  $\implies$ существует сходящаяся подпоследовательность $x_{n_k}$.

    Тогда  $\lim x_{n_k} \eqqcolon b \in K \implies \underbrace{\lim f(x_{n_k})}_{=B} = f(b) \implies f(b) = \sup\limits_{x \in K} f(x) = B \implies f(x) \le f(b) \quad \forall x \in K$.
\end{proof}
%END TICKET 43