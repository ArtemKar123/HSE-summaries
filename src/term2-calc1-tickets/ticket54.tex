%BEGIN TICKET 54
\begin{definition}
    $X$ --- пространство с нормой,  $x_n \in X$.

     $\sum\limits_{k=1}^\infty x_k$ --- ряд. Частичная сумма ряда  $S_n \coloneqq \sum\limits_{k=1}^n x_k$.

     Если  $\exists \lim\limits_{n \to \infty}$, то он называется суммой ряда.

     Ряд сходится, если у него есть сумма (и для $\R$ эта сумма конечна), иначе она бесконечна.
\end{definition}
\begin{theorem}[Необходимое условие сходимости]
    Если ряд $\sum\limits_{n=1}^\infty x_k$ --- сходится, то  $\lim x_n = 0$.
\end{theorem}
\begin{proof}
    $S_n \coloneqq \sum\limits_{k=1}^n x_k \to S \implies \underbrace{S_n - S_{n-1}}_{x_n} \to S - S = 0$.
\end{proof}
\begin{properties}
    \begin{enumerate}
        \item Линейность. $\sum\limits_{n=1}^\infty (\alpha x_n + \beta y_n) = \alpha \sum\limits_{n=1}^\infty x_n + \beta \sum\limits_{n=1}^\infty y_n$.
        \item Расстановка скобок. В ряду произвольным образом можно ставить скобки, то расстановка скобок дает тот же результат. 

            \textbf{Набросок доказательства:} мы просто смотрим на предел подпоследовательности.
        \item В  $\CC$ и  $\R^n$ сходимость равносильна покоординатной сходимости.
    \end{enumerate}
\end{properties}
\begin{theorem}[Критерий Коши]
    $X$ --- полное нормированное пространство.

    Тогда ряд  $\sum\limits_{n=1}^\infty x_n$ сходится  $\iff \forall \eps > 0 \exists N \forall m,n \ge N\!: \| \sum\limits_{k=m}^n x_j \| < \eps$.
\end{theorem}
\begin{proof}
    $S_n \coloneqq \sum\limits_{k=1}^n x_k$. Последовательность  $S_n$ сходится  $\iff S_n$ --- фундаментальная  $\iff \forall \eps > 0 \exists N \forall m, n > N\!: \|S_n-S_m\| < \eps \iff \| \sum\limits_{k=m+1}^n x_k \| < \eps$.
\end{proof}
%END TICKET 54