%BEGIN TICKET 44
\begin{theorem}
    $f\!: X \to Y$ непрерывна во всех точках, биекция и  $X$ --- компакт. Тогда  $f^{-1}$ непрерывна во всех точках.
\end{theorem}
\begin{proof}
    Проверяем непрерывность $f^{-1}$ в терминах открытых множеств. Надо для  $f^{-1}$ проверить , что прообраз открытого --- открыт, то  есть для  $f$ проверить, что образ открытого открыт.

     $U$ --- открыто в  $X \implies X \setminus U$ --- замкнуто и $\subset X$ --- компакт  $\implies X \setminus U$ --- компакт  $\xRightarrow{\text{непрер.}} f(X \setminus U) = Y \setminus f(U)$ --- компакт  $\implies Y \setminus f(U)$ --- замкнуто  $\implies f(U)$ --- открыто.
\end{proof}
\begin{definition}
    $f\!: E \subset X \to Y$ равномерно непрерывна, если  $\forall \eps > 0 \exists \delta > 0 \forall x, y \in E\!: \rho_X(x, y) < \delta \implies \rho_Y(f(x), f(y)) < \eps$.
\end{definition}
\begin{theorem}[Теорема Кантора]
    $f\!: K \to Y$ непрерывна,  $K$ --- компакт. Тогда  $f$ равномерно непрерывна. 
\end{theorem}
\begin{proof}
    Берем $x \in K$,  $f$ непрерывна в точке  $x \implies \exists r_x > 0\!: f(B_{r_x}(x)) \subset B_{\frac{\eps}{2}}(f(x))$. 

    Тогда $K \subset \bigcup\limits_{x \in K} B_{r_x}(x)$ --- открытое покрытие $K$. Возьмем  $\delta > 0$ из леммы Лебега, то есть  $\forall x \in K\ B_\delta(x)$ целиком попал в какой-то элемент покрытия. 

    Проверим, что это  $\delta > 0$ подходит в определение равномерной непрерывности.

     $\forall x, y \in K\ \rho_X(x, y) < \delta \implies y \in B_\delta(x) \implies \exists a \in K\!: B_\delta(x) \subset B_{r_a}(a) \implies x, y \in B_{r_a}(a) \implies f(x), f(y) \in B_{\frac{\eps}{2}}(f(a)) \implies \rho_Y(f(x), f(a)) < \frac{\eps}{2} \land \rho_Y(f(y), f(a)) < \frac{\eps}{2} \implies \rho_Y(f(x), f(y)) < \eps$ по неравенству треугольника.
\end{proof}
%END TICKET 44