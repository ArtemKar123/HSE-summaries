%BEGIN TICKET 53
\begin{theorem}
   $A\!: X \to Y$ --- линейный оператор. Следующие условия равносильны:
   \begin{enumerate}
       \item $A$ --- ограниченный оператор.
       \item  $A$ --- непрерывен в нуле.
       \item  $A$ --- непрерывен во всех точках.
       \item $A$ --- равномерно непрерывен.
   \end{enumerate}
\end{theorem}
\begin{proof}
    $4 \implies 3 \implies 2$ --- очевидно.

    $1 \implies 4$  $\|Ax - Ay\|_Y = \|A(x-y)\|_Y \le \|A\| \cdot \|x-y\|_X$. Если $\|x-y\|_X < \frac{\eps}{\|A\|}$, то $\|Ax - Ay\| < \eps$, а это есть равномерность.

    $2 \implies 1$. Возьмем  $\eps = 1$ и  $\delta > 0$ из определения непрерывности.  $\forall x \in X\!: \|x \| < \delta \implies \|Ax\| < 1$.

    Пусть $\|y\| < 1$. Тогда  $\|\delta y\| < \delta \implies \|A(\delta y)\| < 1 \implies \|Ay\| < \frac{1}{\delta} \implies \sup\limits_{\| y \| < 1} \| Ay\| \le \frac{1}{\delta}$.
\end{proof}
\begin{consequence}
    \begin{enumerate}
        \item $\|Ax\|_Y \le \| A \| \|x\|_X \quad \forall x\in X$.
        \item $\| A B \| \le \|A \| \cdot \|B\|$.
    \end{enumerate}
\end{consequence}
\begin{proof}
    \begin{enumerate}
        \item[2.] $\|A(Bx)\| \le \|A\| \cdot \|Bx\| \le \| A \| \|B\| \|x \|$.

        $\|AB\| = \inf \{c > 0 \mid \|A(Bx)\| \le C\|x\|\} \implies \| AB\| \le \|A\|\|B\|$.

    \item[1.] а где
    \end{enumerate}
\end{proof}
\begin{theorem}
    $A\!: \R^n \to \R^m$.  $A = \begin{pmatrix} a_{11} & \ldots & a_{1n} \\ \vdots & \ddots & \vdots \\ a_{m1} & \ldots & a_{mn} \end{pmatrix}$.

    Тогда $\|A\|^2 \le \sum\limits_{k=1}^n \sum\limits_{j=1}^m a_{jk}^2$. В частности, все такие операторы ограничены.
\end{theorem}
\begin{proof}
    $\|Ax\|^2 = \sum\limits_{j=1}^m \underbrace{\left(\sum\limits_{k=1}^n a_{jk} x_k \right)^2}_{\text{Минковский}} \le\ \text{(Коши-Буняковский)}\ \sum\limits_{j=1}^m \sum\limits_{k=1}^n a_{jk}^2 \underbrace{\sum\limits_{k=1}^n x_k^2}_{=\|x\|^2}$. Следовательно, $\|Ax\| \le \|x\| \sqrt{\sum\limits_{k=1}^m \sum\limits_{j=1}^n a_{jk}^2} \ge \|A\|$.
\end{proof}
\begin{remark}
    В бесконечномерном случае бывают неограниченные операторы.
\end{remark}
%END TICKET 53