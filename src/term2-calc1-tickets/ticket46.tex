%BEGIN TICKET 46
\begin{definition}
    $(X, \rho)$ --- метрическое пространство.  ($\R^d$ --- ключевой случай).

    Непрерывное $\gamma\!: [a,b] \to X$ непрерывное --- путь.

     $\gamma(a)$ --- начало пути,  $\gamma(b)$ --- конец пути.  $\gamma([a, b])$ носитель пути.

     Замкнутый путь  $\gamma(a) = \gamma(b)$. Простой (самонепересекающийся) путь:  $\gamma(u) \neq \gamma(v) \quad \forall u, v \in [a, b]$. Возможно, за исключением равенства  $\gamma(a) = \gamma(b)$.
\end{definition}
\begin{definition}
    Эквивалентные пути: $\gamma_1\!:[a, b] \to X$, $\gamma_2\!: [c,d] \to X$. Если  $\exists u\!:[a, b] \to [c, d]$,  $u$ --- непрерывна и строго монотонно возрастает,  $u(a) = c, u(b) = d$, такой, что  $\gamma_1 = \gamma_2 \circ u$.

    Неформально говоря, мы считаем, что пути эквивалентны, если у них отличается только время прохождения.\\
    (*) $u$ --- допустимое преобразование параметра.
\end{definition}
\begin{definition}
    Класс эквивалентных путей --- кривая.

    Конкретный представитель класса --- параметризация кривой.
\end{definition}
\begin{definition}
    $\gamma\!: [a,b] \to \R^d$.  $r$-гладкий путь, если  $\gamma = \begin{pmatrix} \gamma_1 \\ \gamma_2 \\ \vdots \\ \gamma_d \end{pmatrix}, \gamma_j\!:[a,b] \to \R$ --- $r$-гладкие функции, то есть  $\gamma_j \in C^r[a,b]$.

    Кривая гладкая, если у нее есть гладкая параметризация. Если  $r$ опущено, то  $r=1$.
\end{definition}
%END TICKET 46