\begin{definition}
    Евклидовым пространством называется пара $(V, (\ ,\ ))$, так что $V$ --- векторное пространство над  $\R$.

    $(\ ,\ )\: V \times V \to \R$, такой что  $(\ ,\ )$ билинейна,  $(\ ,\ )$ --- симметрична и  $\forall v \in V\!: (v, v) \ge 0 \land (v, v) = 0 \iff v = 0$.

    Будем называть $(\ ,\ )$ скалярным произведением.
\end{definition}
\begin{example}
    $V=\R^n$. Тогда формула известна. Очев, что все очев.
\end{example}
\begin{example}
    $V = C[0, 1], (f, g) = \int\limits_0^{1}f(x)g(x) \mathrm{d}x$.
\end{example}
\begin{definition}
    $V$ --- евклидово  $v \in V$, норма  $V$ --- $\|v\| = \sqrt{(v, v)}$.

     $v_1, v_2 \in V, d(v_1, v_2) = \|v_1 - v_2\|$.
\end{definition}
\begin{statement}
    $d$ --- метрика.
\end{statement}
\begin{statement}
    $V$ --- евклидово. $v_1, v_2 \in V \implies |(v_1,v_2)|^2 \le \|v_1\| \cdot \|v_2\|$.
\end{statement}
\begin{definition}
    $V$ --- евклидово. Тогда  $\langle v_1, v_2 \rangle$ --- это $\alpha \in [0; \pi]$,  такой, что  $\cos \alpha = \frac{(v_1, v_2)}{\|v_1\| \cdot \|v_2\|}$. 
\end{definition}
\begin{definition}
    Если $(v_1, v_2) = 0$, то будем называть $v_1, v_2$ ортогональными.
\end{definition}
\begin{definition}
    $V$ --- евклидово пространство,  $v_1, \ldots, v_n$ --- базис. Тогда матрица $a_{ij} = (v_i, v_j)$ --- матрица Грама.
\end{definition}

Возьмем $u_1, u_2 \in V$, с координатами $x_i, y_i$ в векторе  $\{v_i\}$. Тогда скалярное  $(u_1, u_2) = u_1Gu_2$, где $G$ --- матрица Грама.

Если матрица Грама равна  $E$, то у нас ортонормированный базис.

\begin{definition}
    Пусть $v_1, \ldots, v_n$ --- базис. Тогда он ортонормирован, если $\forall i,j\!: (v_i, v_j) = \delta_{ij}$.
\end{definition}
\begin{theorem}[оротогонализация Грама Шмидта]
    $V$ --- евклидово пространство,  $v_1, v_2, \ldots, v_n$ --- базис.
    Тогда $\exists$ ортонормированный базис (ОНБ)  $e_1, \ldots, e_n$, причем $\forall i\!: \langle v_1, \ldots, v_i\rangle \overset{(*)}{=} \langle e_1, \ldots, e_i\rangle$.
\end{theorem}
\begin{proof}
    Докажем, что $\exists$ базис со  $(*)$ по индукции.

    База: $e_1 = \frac{v_1}{\|v_1\|}$. $\langle e_1\rangle = \langle v_1\rangle$, $(e_1, e_1) = \frac{1}{\|v_1\|^2} (v_1, v_1) = 1$.

    Переход $l \to l+1$.  Строим  $\widetilde{e_{l+1}} = a_1e_1 + a_2e_2 + \ldots + v_{l+1}, a_i \in \R$. $\langle e_1, \ldots, e_l, \widetilde{e_{l+1}} \rangle = \langle e_1, e_2, \ldots, e_l, v_{l+1} \rangle = \langle v_1, v_2., \ldots, v_{l+1}\rangle$.

    Проверим ортогональность. $\forall a_i\!: (a_1e_1 + \ldots + a_le_l + v_{l+1}, e_i) = 0$i

    Это следует из того, что можно раскрыть сумму $\sum\limits_{j=1}^l a_j(e_j, e_i) + (v_{l+1}, e_i) = 0 \implies a_i(e_i, e_j) + (v_{l+1}, e_i) = 0$. Тут почти все равно нулю, кроме  $(e_i, e_i) = 1$. Тогда  $a_i = - \frac{(v_{l+1}, e_i)}{(e_i, e_i)} = -(v_{l+1}, e_i)$.
\end{proof}
\begin{remark}
    Пусть $e_1, e_2, \ldots, e_n$ --- ОНБ,$v \in V$.  $x_1, \ldots, x_n$ --- координаты $v$. Тогда  $v\cdot e_i = (\sum x_i e_i, e_i) = \sum x_j(e_j, e_i) = x_i$.
\end{remark}

С предыдущей лекции нам известно, что скалярное произведение --- симметрическая билинейная форма над  $\R$.
\begin{example}
    $\R^2, f(\begin{pmatrix} x_1 \\ y_1 \end{pmatrix}, \begin{pmatrix} y_1 \\ y_2 \end{pmatrix}) =x_1y_1 - x_2y_2$ или $(x_1-x_2)(y_1-y_2)$ или $x_1y_2 - x_2y_1$.

    Это все симметрические билинейные формы. 

    Или $V=2^M$ и  $f(x,y) = | X \cap Y| \pmod 2$, если явно записывать, то  $f(\begin{pmatrix} x_1 \\ \vdots \\ x_n \end{pmatrix}, \begin{pmatrix} y_1 \\ \vdots \\ y_n \end{pmatrix}) = x_1y_1 + \ldots + x_ny_n$.
\end{example}
\begin{definition}
    Симметрическая билинейная форма на $V$ (над $K$) называется невырожденной если $(\forall y \in V\!L f(x, y) = 0) \implies x = 0$.
\end{definition}
\begin{definition}
    $V$ --- векторное пространство, то  $V^* = Hom(V, K)$ --- множество линейных отображений (функционалов $V \to K$).
\end{definition}

Пусть $f$ --- билинейная форма  $\implies \exists F\!: V \to V^*, x \mapsto f_x\!: f_x(g) = f(x, y)$.

$f$ --- невырождена $\implies F$ --- изоморфизм. 

Пусть $f$ --- билинейная форма  $\leadsto \Gamma_f = (a_{ij})$,  $a_{ij} = f(e_i, e_j)$, где  $\{e_i\}$ --- базис. Тогда  $f(\sum x_i e_i, \sum y_i e_i) = \sum a_{ij} x_i y_hj = X^T \Gamma_f Y$.

Перейдем теперь  $K = \CC$.  $f(\begin{pmatrix} x_1 \\ \vdots \\ x_n \end{pmatrix}, \begin{pmatrix} y_1 \\ \vdots \\ y_n \end{pmatrix}) = \sum x_i \overline{y}_i$, $f(x, x) > 0$.

\begin{definition}
    Унитарным пространством называется пара $(V, (-,-))$, где  $V$ --- векторное пространство над  $\CC$,  $(-, -)\!: V \times V \to \CC$, такое, что 
     \begin{enumerate}
     \item $(x_1 + x_2, y) = (x_1, y) + (x_2, y)$ или $(x, y_1 + y_2) = (x, y_1) + (x, y_2)$.
     \item $(ax, y) = a(x, y), (x, ay) = \overline{a}(x, y)$.
     \item $f(x, y) = \overline{f(y, x)}$.
     \item  $f(x, x) \in \R_+$, при  $x \neq 0$.
    \end{enumerate}
    Длина, КБШ, метрика --- все тоже самое.
\\
    С углами сложнее (это останется тайной).
    \\
    Отрогонализация Грамма-Шмидта работает, в чатсности, $\forall$ унитарном пространстве есть ортонормированный базис. При этом $f(X, Y) = X^T\Gamma_f\overline{Y}$.
\end{definition}
\Subsection{Ортогональное дополнение}
 \begin{definition}
     $V$ --- эвклидово/унитарное пространство:  $U \le V$. Ортогональное дополнение $U$ --- это $U^\perp = \{ v \in V \mid (u,v) = 0 \forall u \in U\}$.
 \end{definition}
 \begin{statement}
     $U^\perp$ --- подпространство в  $V$.
 \end{statement}
 \begin{proof}
     Упражнение.
 \end{proof}
 \begin{theorem}
     \slashn
      \begin{enumerate}
          \item $\dim U^\perp = \dim V - \dim U$.
          \item  $U^{\perp\perp} = U$.
          \item  $V = U \oplus U^\perp$ --- любой элемент из $V$ представим в виде пары из  $U$ и  $U^\perp$.
     \end{enumerate}
 \end{theorem}
 \begin{proof}
     \slashn
     \begin{enumerate}
         \item $\dim V = n, \dim U = K$. Возьмем базис $e_1, \ldots, e_k$, дополним через $e_{k+1}, \ldots, e_n$ до ОНБ (до любого, а далее Грамм-Шмидт).

             Заметим, что $x = \sum x_ie_i \in U^{\perp} \iff (\sum\limits_{i=1}^n x_ie_i, e_j) = 0 \forall j=1,..,k \land e_j \in U^{\perp} \iff x_j = 0 \forall j=1..k \iff x = \sum\limits_{k+1}^n x_ie_i \iff x \in \langle e_{k+1}, \ldots, e_n \rangle$.

            Итого, $\dim U^\perp = n - k = \dim V - \dim U$.

            Тогда  $U^{\perp\perp} = \ldots$. 

            Ну и понятно, как раскладывается вектор из $V$.
     \end{enumerate}
 \end{proof}
 \begin{definition}
     $v = u+u^{\perp}, u \in U, u^{\perp} \in U^{\perp}$,  $u$ --- проекция  $v$ на  $U$, $u^\perp$ --- ортогональная составляющая.

     $\|u^{\perp}\|$ --- расстояние от $v$ до  $u$.
 \end{definition}
 %\begin{remark}
     %Для любых билинейных форм верно $U \subset U^{\perp\perp}$.
 %\end{remark}
 \begin{statement}[Матрица Грама при замене базиса]
     $f$ --- билинейная полуторномерная форма.  $\Gamma_f$ --- матрица Грама в базисе  $e_1, \ldots, e_n$.

     Возьмем другой базис, $C$ --- матрица перехода.

     $X \leadsto C X = \widetilde{X}$. Пусть  $D = C^{-1}$. Тогда  $X = D\widetilde{X}$.

     $f(x, y) = X^T\Gamma Y = \ldots = \widetilde{X}^{T} (D^T\Gamma D)\widetilde{Y} \implies D^T\Gamma D = \widetilde{\Gamma}$ --- матрица Грама в новом базисе.
 \end{statement}
 \Subsection{Соответствиемежду формами/матрицами}

 Билинейная полуторная форма $\implies A \in M_n(K)$.

 Симметрическая билинейная форма  $\iff A = A^T$ --- симм. матрица. ($A = \Gamma_f$).

 Полуторная форма $\iff a_{ij} = \overline{a_{ji}} \iff \overline{A^T} = A$,  $A^* \coloneqq \overline{A^T}$ матричное сопряжение.

  $A = A^*$ --- эрмитова матрица.

 $f$ --- симметрическая билинейная/полуторная форма.

 Переформулировка: дана  $f(X, X) = \sum a_{ij} x_i x_j$ --- квадратичная формат.

 Как понять: верно ли, что  $q(x) > 0 \ \forall x \neq 0$.

  $f$ -- положительно определена  $\implies f$ --- скалярное произведение  $\implies \exists $ ОНБ:  $\exists C\!:$  $\det C^TAC = \det(C^T)\det(C)\det(A) = (\det C)^2 \det A = 1 = \det E \implies \det A > 0$

  \begin{theorem}[Критерий Сильвестера]
    $f$ --- симметричная билинейная,  $A \in M_n(\R)$ --- матрица Грама  $f$  в базисе $e_1, \ldots, e_n$.

    Тогда $f$ --- полодительно определена  $\iff \forall i = 1..n \vartriangle_i > 0$, где $\vartriangle_i \coloneqq \det(a_{jk})_{\substack{j=1..i\\k=1..i}}$.
\end{theorem}
\begin{proof}
    Необходимость. $I = \{i_1, i_2, \ldots, i_k\}$ --- номера строк/столбцов.
\\
    $A_I$ --- подматрица со строками/столбцами из  $I$.  $A_I$ --- матрица Грама.  $f\Big|_{\langle e_{i_1}, \ldots, e_{i_k} \rangle}$ --- положительная определена.

    Достаточность. Пусть $\Delta_i > 0$. Докажем индукцией по $k$:  $f\Big|_{\langle e_1, \ldots, e_k\rangle}$ ---  положительно определена $\implies$ при  $k = n$ то , что нажо.

    База:  $f \Big|_{\langle e_1 \rangle}\ f(ae_1, ae_1) = a^2(e_1, a) = a^2a_1 > 0$.

    Переход: $k \to k + 1$.  $f\Big|_{\langle e_1, \ldots, e_k \rangle}$ --- положительно определена, $()\langle e_1, \ldots, e_k \rangle, f)$ --- евклидово пространство. $\implies \exists$ ОНБ, $\widetilde{e_1}, \ldots, \widetilde{e_k}$. Матрица Грама $f \Big|_{\langle e_1, \ldots, e_{k+1} \rangle}$ в базисе $\widetilde{e_1}, \ldots, \widetilde{e_k}, e_{k+1} =
    \left(\begin{array}{ccc|c}
    1 & \ldots & 0 & a_1\\
    \vdots & \ddots & \vdots & \vdots\\
    0 & \ldots & 1 & a_k\\ \hline
    a_1 & \ldots & a_k & a
\end{array}\right)$. 
$\widetilde{e}_{k+1} = e_{k+1} \sum\limits_{i=1}^k a_i\widetilde{e_i}$. Теперь $\widetilde{e}_{k+1}, \widetilde{e}_i = (e_{k+1} - \sum a_i \widetilde{e_i}, \widetilde{e}_i) = (e_{k+1}, \widetilde{e}_i) - \sum\limits_{j=1}^k a_j(\widetilde{e_j}, \widetilde{e_i}) = a_i - a_i = 0$.

Тогда $\widetilde{A} = 
\begin{pmatrix}
    1 & \ldots & 0\\
    \vdots & \ddots & \vdots\\
    0 & \ldots & b
\end{pmatrix}$, $\det \widetilde{A} = \det (C^T) \cdot \Delta_{k+1} \det C > 0 \implies b > 0 \implies f$ имеет ОНБ в $\langle e_1, \ldots, e_{k+1} \rangle \implies f$ --- положительно определена на $\langle e_1, \ldots, e_{k+1} \rangle$.
\end{proof}
\Section{Операторы в евклидовых и унитарных пространствах}{ХБ}
\Subsection{Самосопряженные операторы}
\begin{definition}
    $V$ --- евклидово/унитарное пространство,  $\CA \in End(V)$.

    $\mathcal{B}$ --- сопряженный оператор, если  $(\CA X, y) = (X, \mathcal{B}y) \forall x, y \in V$.
\end{definition}
\begin{theorem}
    $\exists \land \exists!$ сопряженный оператор. $\CA = \CA^*$ --- называется самосопряженным.
\end{theorem}
\begin{theorem}
    $\CA$ --- самосопряжен  $\iff$ у  $\CA$ есть ОНБ из собственных векторов и все с.ч вещественны. 
\end{theorem}

